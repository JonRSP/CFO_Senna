%% ==================================================================
%% Documento baseado na classe abntex2CBMSC.cls
%% ==================================================================
%%
%% Este documento usa a classe abntex2CBMSC, que é uma adaptação da
%% classe abntex2.cls. A adaptação foi realizada para atender às
%% necessidades do Corpo de Bombeiros Militar de Santa Catarina (CBMSC).
%%
%% Autor da adaptação: Jonatas Senna <jonatas.senna.bmsc@gmail.com.com>
%% Instituição: CBMSC
%% Data: 06/02/2025
%%
%% Licença da classe: LaTeX Project Public License (LPPL) v1.3c ou posterior
%% Consulte: https://www.latex-project.org/lppl.txt
%%
%% Este documento pode ser modificado livremente para atender às 
%% necessidades dos usuários, desde que respeitada a licença da classe.
%%
%% ==================================================================

% ------------------------------------------------------------------------
% ------------------------------------------------------------------------
% abnTeX2CBMSC: trabalhos monograficos em geral em conformidade com 
% ABNT NBR 14724:2011: Informacao e documentacao - Trabalhos academicos
% ------------------------------------------------------------------------
% ------------------------------------------------------------------------
% ---
% Classe do Documento, não alterar nada
% ---
\documentclass[
	% -- opções da classe memoir --
	12pt,				% tamanho da fonte
	oneside,			% para impressão em recto e verso. Oposto a oneside
	a4paper,			% tamanho do papel. 
	% -- opções da classe abntex2 --
	chapter = TITLE,		% títulos de capítulos convertidos em letras maiúsculas
	section=TITLE,		% títulos de seções convertidos em letras maiúsculas
	subsection=TITLE,	% títulos de subseções convertidos em letras maiúsculas
	subsubsection=TITLE,% títulos de subsubseções convertidos em letras maiúsculas
	% -- opções do pacote babel --
	english,			% idioma adicional para hifenização
	brazil				% o último idioma é o principal do documento
	]{SRC/abntex2CBMSC}


% ---
% Informações de dados básicos
% ---
\curso{CFO} % CFO, CCEM ou CAEE

\titulo{Uma Abordagem Experimental para Avaliação de Uniformes Operacionais: Um Estudo Aplicado à Busca Terrestre} %Título do trabalho

\autor{Jonatas Ribeiro Senna Pires} %Nome do autor

\local{Florianópolis} %Cidade

\ano{2025} % Ano de apresentação do trabalho
\mes{Setembro} % Mês da apresentação do Trabalho
\dia{12} % Dia da apresentação do Trabalho

\instituicao{Corpo de Bombeiros Militar de Santa Catarina} %nome da instituição

\linhaDePesquisa{Tecnologia na Atividade BM: estuda a direção, o preparo e uso de tecnologias, equipamentos e materiais} % linha de pesquisa segundo a ig40


% ---
% Informações da banca
% até 5 membros no total está bem configurado
% ---
\orientador[\posto{MAJ} \corporacao{BM}]{Renan César Vinotti Ceccato}{CBMSC} % \Orientador[título]{Nome}{corporação}

%\coorientador[\posto{MAJ} \corporacao{BM}]{Natália Cauduro}{CBMSC}

\adicionarMembro{\posto{CAP} \corporacao{BM}}{Wagner Alberto de Moraes}{CBMSC} %membro da banca 
\adicionarMembro{\posto{MAJ} \corporacao{BM}}{Alan Delei Cielusinsky}{CBMSC}


% ---
% Palavras Chave
% --- 
\adicionarPalavraChave{Fardamento Operacional} %palavras chave em português
\adicionarPalavraChave{Busca Terrestre}
\adicionarPalavraChave{EPI}
\adicionarPalavraChave{Mobilidade}

\adicionarKeyword{Operational Uniform} %palavras chave em inglês
\adicionarKeyword{Terrestrial Search}
\adicionarKeyword{PPE}
\adicionarKeyword{Mobility}



% ---
% arquivo com definições gerais, pacotes e funções que geram os elementos pré textuais
% ---


% ---
% Pacotes básicos 
% ---
\usepackage{times}			   % Usa a fonte times new roman			
\usepackage[T1]{fontenc}		% Selecao de codigos de fonte.
\usepackage[utf8]{inputenc}		% Codificacao do documento (conversão automática dos acentos)
\usepackage{indentfirst}		% Indenta o primeiro parágrafo de cada seção.
\usepackage{color}				% Controle das cores
\usepackage{graphicx}			% Inclusão de gráficos
\graphicspath{ {./Imagens/} }
\usepackage{microtype} 			% para melhorias de justificação
\usepackage[acronym,nonumberlist]{glossaries}
\usepackage{csquotes}
\usepackage{enumitem}% http://ctan.org/pkg/enumitem
\usepackage{pdflscape}
\usepackage{tabularx}
\usepackage[a4paper]{geometry}
\usepackage{adjustbox}
\usepackage{gensymb}


% ---
% Pacotes de citações
% ---
\usepackage[backend=biber,style=abnt,backref=true]{biblatex}
\addbibresource{referencias.bib} 



% ---




% informações do PDF
\makeatletter
\hypersetup{
     	%pagebackref=true,
		pdftitle={\@title}, 
		pdfauthor={\@author},
    	pdfsubject={\imprimirpreambulo},
	    pdfcreator={LaTeX with abnTeX2},
		pdfkeywords={abnt}{latex}{abntex}{abntex2}{trabalho acadêmico}, 
		colorlinks=true,       		% false: boxed links; true: colored links
    	linkcolor=black,          	% color of internal links
    	citecolor=black,        		% color of links to bibliography
    	filecolor=magenta,      		% color of file links
		urlcolor=blue,
		bookmarksdepth=4
}
\makeatother
% --- 
% Verifica se há figuras no documento

% ---
% Posiciona figuras e tabelas no topo da página quando adicionadas sozinhas
% em um página em branco. Ver https://github.com/abntex/abntex2/issues/170
\makeatletter
\setlength{\@fptop}{5pt} % Set distance from top of page to first float
\makeatother
% ---

% ---
% Possibilita criação de Quadros e Lista de quadros.
% Ver https://github.com/abntex/abntex2/issues/176
%
\newcommand{\quadroname}{Quadro}
\newcommand{\listofquadrosname}{Lista de quadros}

\newfloat[chapter]{quadro}{loq}{\quadroname}
\newlistof{listofquadros}{loq}{\listofquadrosname}
\newlistentry{quadro}{loq}{0}

% configurações para atender às regras da ABNT
\setfloatadjustment{quadro}{\centering}
\counterwithout{quadro}{chapter}
\renewcommand{\cftquadroname}{\quadroname\space} 
\renewcommand*{\cftquadroaftersnum}{\hfill--\hfill}


\setfloatlocations{quadro}{hbtp} % Ver https://github.com/abntex/abntex2/issues/176
% ---
\DeclareLabelalphaTemplate{
  \labelelement{
    \field[strwidth=1]{year}
  }
  \labelelement{
    \literal{a}
  }
}

\renewbibmacro*{author}{%
  \ifboolexpr{ test {\ifcitation} }
    {\ifnameundef{shortauthor} % Se for uma citação, usa shortauthor
      {\printnames{author}}
      {\printnames{shortauthor}}}
    {\printnames{author}} % Se for uma referência completa, usa author
}
% --- 
% Espaçamentos entre linhas e parágrafos 
% --- 

% O tamanho do parágrafo é dado por:
\setlength{\parindent}{1.3cm}

% Controle do espaçamento entre um parágrafo e outro:
\setlength{\parskip}{0.2cm}  % tente também \onelineskip

% ---
% compila o indice
% ---
\makeindex
% ---

% ----
% Início do documento
% ----
\usepackage{float}
\makeglossaries
\usepackage{newtxtext,newtxmath}



\begin{document}
\newgeometry{bottom=2cm}
% Arquivo com as definições de siglas

\sigla{IoT}{Internet das Coisas}
\sigla{CBMSC}{Corpo de Bombeiros Militar de Santa Catarina}
\sigla{DS}{Dispositivo Sensitivo}
\sigla{MAE}{Mean Absolute Error}
\sigla{MAPE}{Mean Absolute Percentage Error}
\sigla{MQTT}{Message Queuing Telemetry Transport}
\sigla{FMS}{\textit{Functional Movement Screen}}
\sigla{BTR}{Busca Terrestre}
\sigla{EPI}{Equipamento de Proteção Individual}
\sigla{MML}{MultiMissão Leve}
\sigla{MM}{MultiMissão}
\sigla{RVE}{Resgate Veicular}
\sigla{EFM}{Educação Física Militar}
\sigla{PMSC}{Polícia Militar de Santa Catarina}
\sigla{CV}{Coeficiente de Variação}
% Importando as siglas do arquivo externo



% Seleciona o idioma do documento (conforme pacotes do babel)
%\selectlanguage{english}
\selectlanguage{brazil}

% Retira espaço extra obsoleto entre as frases.
\frenchspacing 

% ----------------------------------------------------------
% ELEMENTOS PRÉ-TEXTUAIS
% ----------------------------------------------------------
% \pretextual

% ---
% Capa
% ---
\imprimircapa
% ---

% ---
% Folha de rosto
% (o * indica que haverá a ficha bibliográfica)
% ---
\imprimirfolhaderosto*
% ---

\imprimirFichaCatalografica


% ---

% ---
% Inserir errata
% ---

% ---

% ---
% Inserir folha de aprovação
% ---

\newpage
\imprimirFolhadeAprovacao


% ---

\imprimirDedicatoria
% ---

% ---
% Agradecimentos
% ---
 \newpage
 \imprimirAgradecimentos

% ---

% ---
% Epígrafe
% ---
\imprimirEpigrafe
% ---

% ---
% RESUMOS
% ---

% resumo em português
\imprimirResumo
  


% % resumo em inglês
\imprimirAbstract

% ---

% ---
% inserir lista de ilustrações
% ---
\IfFileExists{imagem.txt}{
\newpage
    \pdfbookmark[0]{\listfigurename}{lof} % Adiciona bookmark
    \listoffigures* % Gera a lista de figuras
}{}
% ---

% ---
% inserir lista de quadros
% ---
%\pdfbookmark[0]{\listofquadrosname}{loq}
%\listofquadros*
%\cleardoublepage
% ---

% ---
% inserir lista de tabelas
% ---
\IfFileExists{tabela.txt}{%
\newpage
\listoftables*
}{}

\cleardoublepage

\listadeacronimos

% ---
% inserir o sumario
% ---
\pdfbookmark[0]{\contentsname}{toc}

\renewcommand{\contentsname}{%
  \centering\textbf{SUMÁRIO}%
}

\tableofcontents*
\cleardoublepage
% ---



% ----------------------------------------------------------
% ELEMENTOS TEXTUAIS
% ----------------------------------------------------------


% ----------------------------------------------------------
% Introdução
% ----------------------------------------------------------

\input{Texto/introducao}

% ----------------------------------------------------------
% Importa os capítulos de forma automática quando o arquivo é compilado pelo comando fornecido
% ----------------------------------------------------------

\input{capitulos.tex}

% ----------------------------------------------------------
% Conclusão
% ----------------------------------------------------------
\input{Texto/conclusao}

% ----------------------------------------------------------
% Finaliza a parte no bookmark do PDF
% para que se inicie o bookmark na raiz
% e adiciona espaço de parte no Sumário
% ----------------------------------------------------------
\phantompart


% ----------------------------------------------------------
% ELEMENTOS PÓS-TEXTUAIS
% ----------------------------------------------------------
\postextual
% ----------------------------------------------------------

% ----------------------------------------------------------
% Referências bibliográficas
% ----------------------------------------------------------
% \bibliography{abntex2-modelo-references}

\newpage

\defbibheading{bibliography}[\refname]{%
  \section*{\centering\textbf{\MakeUppercase{#1}}} % Centraliza, textbf e maiúsculo
}
% Aumentando o espaçamento entre itens da bibliografia
\setlength{\bibitemsep}{0.6\baselineskip}
\printbibliography


% ----------------------------------------------------------
% Apêndices
% ----------------------------------------------------------

% ---
% Inicia os apêndices
% ---
\imprimirapendices
% ---


% ----------------------------------------------------------
% Anexos
% ----------------------------------------------------------

% ---
% Inicia os anexos
% ---
\imprimiranexos

%---------------------------------------------------------------------
% INDICE REMISSIVO
%---------------------------------------------------------------------
\phantompart
\printindex
%---------------------------------------------------------------------

\end{document}
