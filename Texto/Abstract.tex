The terrestrial search activity conducted by the Military Fire Department of Santa Catarina (CBMSC) presents operational 
challenges that require uniforms compatible with adverse environmental conditions, extended periods of deployment, and 
high demands for functional mobility. This study aimed to compare three uniforms applicable to the activity: the standart 
operational uniform, the MultiMission PPE (Personal Protective Equipment), and the Lightweight MultiMission PPE. The adopted methodology included the 
application of the Functional Movement Screen (FMS) protocol, temperature data collection using validated sensors, and a 
subjective evaluation by the participants. The results revealed significant differences among the uniforms in terms of 
movement restriction and thermal comfort, highlighting the strengths and weaknesses of each. Although the Lightweight MultiMission 
PPE demonstrated advancements in specific aspects, considering the standardization criteria for PPE comparison, the standart 
operational uniform was classified as the most suitable for the activity. In light of potential future changes to the
operational uniform, it is recommended that this methodology be applied to prototype evaluations. Moreover, the integrated 
analysis of the data emphasizes the need for technical adjustments and the continuous development of Personal Protective 
Equipment to meet the specific demands of terrestrial search operations and related fields.