\chapter{Conclusão}
    O presente estudo comparou os fardamentos operacionais utilizados pelo \acrshort{CBMSC} para a 
    atividade de \acrlong{BTR}, considerando três aspectos fundamentais: mobilidade funcional, 
    comportamento térmico e percepção subjetiva dos usuários. A análise incluiu os fardamentos 5ºA, 
    EPI \acrlong{MM} e EPI \acrlong{MML}, com base em testes padronizados e medições objetivas.
    
    \section{Classificação dos Fardamentos}
        Apresenta-se a seguir a classificação entre os fardamentos em cada uma das categorias.
        \subsection{\textit{Functional Movement Screen}}
            \begin{enumerate}[label=\Roman*.] % Usa números romanos maiúsculos
                \item O 5ºA apresentou os resultados mais próximos à linha de base, não atrapalhando a 
                movimentação.
                \item O \acrshort{EPI} \acrlong{MML} demonstrou mobilidade intermediária, com alguma
                restrição em certos padrões de movimento.
                \item O \acrshort{EPI} \acrlong{MM} apresentou o maior impacto na mobilidade, sendo 
                mais rígido em alguns pontos e com menor flexibilidade nas articulações.
                \end{enumerate}

        \subsection{Comportamento Térmico}
            \begin{enumerate}[label=\Roman*.] % Usa números romanos maiúsculos
                \item O \acrshort{EPI} \acrlong{MM} apresentou os melhores resultados médios em termos de 
                comportamento térmico, apesar de possuir o maior \acrlong{CV}, sugerindo maior dependência 
                sobre a fisiologia do usuário.
                \item O 5ºA demonstrou comportamento térmico intermediário, apresentando uma variação de 
                temperatura elevada em sua parte inferior e menor variação em sua parte superior, mas por 
                ter apresentado o menor \acrlong{CV} de toda a amostra, se destaca positivamente por ter 
                um comportamento mais estável.
                \item O \acrshort{EPI} \acrlong{MML} apresentou o pior comportamento térmico considerando os 
                três fardamentos testados, apresentando uma variação de temperatura menor em 
                sua parte inferior e maior variação em sua parte superior, além de 
                ter apresentado um \acrlong{CV} intermediário.
                \end{enumerate}    

        \subsection{Percepção Subjetiva}
            \begin{enumerate}[label=\Roman*.] % Usa números romanos maiúsculos
                \item O \acrshort{EPI} \acrlong{MML} apresentou a melhor avaliação subjetiva dos usuários, 
                sendo o fardamento com menos avaliações negativas e menos pontos de incômodo.
                \item O 5ºA demonstrou uma avaliação subjetiva intermediária, apresentando críticas 
                pontuais tanto na parte inferior quanto na parte superior do fardamento.
                \item O \acrshort{EPI} \acrlong{MM} apresentou a pior avaliação subjetiva 
                com a maior quantidade de relatos no campo descritivo e a maior incidência de críticas em 
                pontos específicos, em ambas as partes do fardamento. Destacam-se os joelhos e os ombros.
                \end{enumerate}

        Para cada um dos elementos avaliados, atribuiu-se a pontuação 3 para o fardamento melhor classificado, 
        2 para o intermediário e 1 para o que menos se destacou. A pontuação final está descrita 
        na tabela \ref{tab:classificacao}:

        \begin{table}[H]
            \centering
            \caption{Pontuação geral dos fardamentos}
            \label{tab:classificacao}
            \footnotesize
            \begin{tabular}{lrrrr}
            \toprule
            Fardamento               &     FMS    &  Comportamento Térmico &  Percepção Subjetiva &  Pontuação Final \\
            \midrule
            5º A             &  3         &     2                    &       2 &     7             \\
            MultiMissão Leve &  2         &     1                    &       3 &     6             \\
            MultiMissão      &  1         &     3                    &       1 &     5             \\
            \bottomrule
            \end{tabular}
            \end{table}

        Considerando todos os critérios avaliados e a pontuação atribuída a cada um dos uniformes, 
        conclui-se que, dos fardamentos testados e com a metodologia utilizada, o fardamento 
        mais indicado para a atividade de \acrlong{BTR} é o \textbf{fardamento 5ºA}.

    \section{Considerações Finais}
        A pesquisa demonstrou que, embora existam avanços significativos no desenvolvimento de 
        uniformes mais adaptados às necessidades operacionais, nenhum dos modelos analisados 
        atende completamente a todas as demandas da atividade.

        \subsection{O Fardamento MultiMissão Leve}
            Ao se comparar com o \acrlong{MM}, o fardamento \acrlong{MML} mostra uma clara 
            evolução em termos de mobilidade e percepção do usuário. Entretanto ainda carece de 
            melhorias, em especial quando se trata de comportamento térmico.

            A metodologia aplicada pode ter influenciado os resultados, pois a exigência do uso 
            padronizado da jaqueta pode ter prejudicado a avaliação realista do \acrshort{EPI} \acrshort{MML}.
            Apesar do uso do fardamento sem a jaqueta ser admitido, sendo utilizada a calça do \acrshort{EPI} 
            pareada com uma camiseta de manga longa vermelha em 100\% poliamida, o que, a princípio, 
            reduziria os efeitos térmicos da parte superior do uniforme, deve ser considerada sua 
            influência na segurança oferecida e nos impactos na identidade visual dos militares 
            do \acrshort{CBMSC}.

            Um diferencial do \acrshort{MML} em relação aos demais uniformes, que deve ser 
            considerado ao se examinar a possibilidade de institucionalização de um \acrshort{EPI} 
            para determinada atividade, é sua impermeabilidade. Essa característica é imprescindível 
            para atividades como a \acrlong{BTR}, dada sua atuação em ambientes abertos e sujeitos a 
            intempéries.
        
        \subsection{Escolha dos Participantes}
            A seleção dos participantes e o número reduzido de sujeitos podem ter influenciado os 
            resultados, uma vez que todos pertenciam ao mesmo grupo e compartilhavam rotinas 
            semelhantes de treinamento físico e alimentação, o que pode ter introduzido um viés nos achados.

            Além disso, a limitação de tamanho disponível do fardamento \acrshort{MML} restringiu o número de 
            participantes aptos a utilizá-lo, reduzindo a diversidade da amostra e, consequentemente, 
            a capacidade de generalização dos resultados para a população total de bombeiros militares.
        
    \section{Trabalhos Futuros}
        Com base nas limitações e descobertas deste estudo, recomenda-se que futuras pesquisas:

        \begin{itemize}
            \item Expandam a amostra de participantes, incluindo bombeiros de diferentes biotipos, 
            níveis de experiência operacional e de outras regiões do estado.
            \item Apliquem a metodologia de avaliação à futura versão do 5ºA, que está em desenvolvimento, 
            para verificar se as novas modificações impactam positivamente o desempenho operacional.
            \item Apliquem a metodologia de avaliação às combições do \acrshort{MML} com a camiseta manga 
            longa em 100\% poliamida, \acrshort{MM} com a camiseta padrão \acrshort{CBMSC} e o 5ºA com 
            a camiseta padrão \acrshort{CBMSC}.
            \item Utilizem medições fisiológicas mais detalhadas, como monitoramento de frequência 
            cardíaca e termografia, para validar objetivamente os efeitos térmicos de cada fardamento no usuário.
        \end{itemize}