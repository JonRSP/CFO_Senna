\chapter{Conclusão}
    O presente estudo comparou os fardamentos operacionais utilizados pelo \acrshort{CBMSC} para a 
    atividade de \acrlong{BTR}, considerando três aspectos fundamentais: mobilidade funcional, 
    comportamento térmico e percepção subjetiva dos usuários. A análise incluiu os fardamentos 5ºA, 
    EPI \acrlong{MM} e EPI \acrlong{MML}, com base em testes padronizados e medições objetivas.
    
    \section{Classificação dos Fardamentos}
        Apresenta-se a seguir a classificação entre os fardamentos em cada uma das categorias.
        \subsection{\acrlong{FMS}}
            \begin{enumerate}[label=\Roman*.] % Usa números romanos maiúsculos
                \item O 5ºA apresentou os resultados mais próximos à linha de base, não atrapalhando a 
                movimentação.
                \item O \acrshort{EPI} \acrlong{MML} demonstrou mobilidade intermediária, com 
                restrições em alguns padrões de movimento.
                \item O \acrshort{EPI} \acrlong{MM} apresentou o maior impacto na mobilidade, sendo 
                mais rígido em alguns pontos e com menor flexibilidade nas articulações.
                \end{enumerate}

        \subsection{Comportamento Térmico}
            \begin{enumerate}[label=\Roman*.] % Usa números romanos maiúsculos
                \item O \acrshort{EPI} \acrlong{MM} apresentou os melhores resultados médios em termos de 
                comportamento térmico, apesar de possuir o maior \acrlong{CV}, sugerindo maior dependência 
                sobre a fisiologia do usuário.
                \item O 5ºA demonstrou comportamento térmico intermediário, apresentando uma variação de 
                temperatura elevada em sua parte inferior e menor variação em sua parte superior, além de 
                ter apresentado o menor \acrlong{CV} de toda a amostra.
                \item O \acrshort{EPI} \acrlong{MML} apresentou o pior comportamento térmico considerando os 
                três fardamentos testados, apresentando uma variação de temperatura menor em 
                sua parte inferior e maior variação em sua parte superior, além de 
                ter apresentado um \acrlong{CV} intermediário.
                \end{enumerate}    


        \subsection{Percepção Subjetiva}
            \begin{enumerate}[label=\Roman*.] % Usa números romanos maiúsculos
                \item O \acrshort{EPI} \acrlong{MML} apresentou a melhor avaliação subjetiva dos usuários, 
                sendo o fardamento com menos avaliações negativas e menos pontos de incômodo.
                \item O 5ºA demonstrou uma avaliação subjetiva intermediária, apresentando críticas 
                uniformes tanto na parte inferior quanto na parte superior do fardamento.
                \item O \acrshort{EPI} \acrlong{MM} apresentou a pior avaliação subjetiva 
                com a maior quantidade de relatos no campo descritivo e a maior incidência de críticas, 
                em ambas as partes do fardamento, nota-se, em especial os joelhos e os ombros.
                \end{enumerate}