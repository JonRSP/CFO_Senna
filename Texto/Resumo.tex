A atividade de busca terrestre conduzida pelo Corpo de Bombeiros Militar de Santa Catarina (CBMSC) impõe desafios operacionais 
que demandam um fardamento compatível com condições ambientais adversas, longos períodos de atuação e exigência de mobilidade 
funcional. Este trabalho teve como objetivo comparar três uniformes aplicáveis à atividade: o fardamento 5ºA, o EPI MultiMissão 
e o EPI MultiMissão Leve. A metodologia adotada incluiu a aplicação do protocolo Functional Movement Screen (FMS), a coleta de 
dados de temperatura com sensores validados e uma avaliação subjetiva dos participantes. Os resultados revelaram diferenças 
significativas entre os fardamentos quanto à restrição de movimentos e ao conforto térmico, evidenciando os pontos fortes e fracos 
de cada um deles. Embora o EPI MultiMissão Leve tenha apresentado avanços em aspectos específicos, considerando os critérios para padronização na comparação entre os EPIs, o fardamento 
5ºA foi classificado como o mais adequado para a atividade. Levando-se em conta futuras alterações no fardamento operacional, 
recomenda-se a execução da metodologia com os protótipos. Além disso, a análise integrada dos dados evidencia a necessidade de ajustes 
técnicos e desenvolvimento contínuo dos Equipamentos de Proteção Individual, visando atender às demandas particulares da área 
de Busca Terrestre e demais áreas.