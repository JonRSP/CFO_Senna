Este estudo apresenta uma abordagem metodológica experimental para avaliar uniformes operacionais 
utilizados pelo Corpo de Bombeiros Militar de Santa Catarina (CBMSC). Foram comparados três uniformes 
operacionais aplicáveis à área de Busca Terrestre: o fardamento 5ºA, o EPI MultiMissão e a proposta de 
EPI MultiMissão Leve. A metodologia integrou três pilares principais: avaliação da mobilidade 
funcional (utilizando o protocolo Functional Movement Screen - FMS), análise térmica (por meio de 
sensores especialmente desenvolvidos para este trabalho e validados em laboratório) e percepção 
subjetiva dos usuários (através de questionários estruturados). Os resultados indicaram diferenças 
significativas entre os uniformes, especialmente no que se refere à mobilidade e ao conforto térmico. 
O fardamento 5ºA apresentou melhor desempenho geral, embora o EPI MultiMissão Leve tenha demonstrado 
vantagens específicas em mobilidade e conforto. A metodologia proposta neste estudo 
podem contribuir para decisões ainda mais fundamentadas sobre o uso e desenvolvimento de 
uniformes operacionais, reforçando a importância de metodologias integradas e replicáveis para 
futuras avaliações no CBMSC.