
A realização deste Trabalho de Conclusão de Curso não teria sido possível sem o apoio e a colaboração de diversas pessoas, às quais registro meus sinceros agradecimentos.

Ao Coronel BM Diogo Bahia Losso, pela gentileza e disponibilidade, contribuindo de forma inestimável para o resgate histórico e institucional do fardamento 5ºA no CBMSC. Suas palavras e relatos trouxeram profundidade e autenticidade à pesquisa.

Ao Major BM Renan César Vinotti Ceccato, orientador deste trabalho, pela excelência na condução da orientação, pela confiança depositada, pela clareza técnica e pelo incentivo contínuo durante todas as etapas deste estudo.

À Major BM Natália Cauduro, pela valiosa colaboração na etapa de coleta de dados, seu apoio técnico colaborou sobremaneira ao desenvolvimento deste projeto.

Aos meus colegas de turma, os cadetes Régis, Schio, Schlegel, João Pedro, Maurício, Fuck, Medeiros e Teixeira, que aceitaram participar ativamente da fase experimental do trabalho. Sua dedicação e seriedade durante os testes foram fundamentais para a obtenção dos dados e para o êxito desta pesquisa.

À equipe da academia Shark, por gentilmente terem cedido seu espaço e seus equipamentos para a realização do estudo.

Por fim, à minha noiva Amanda, por todo o carinho, compreensão e apoio durante esse processo. Obrigado por compreender minhas ausências, por me incentivar nos momentos de dificuldade e por estar ao meu lado, mesmo quando o cansaço parecia maior que a vontade. Seu apoio foi um alicerce essencial para que eu pudesse concluir mais esta etapa.

A todos, minha mais profunda gratidão.