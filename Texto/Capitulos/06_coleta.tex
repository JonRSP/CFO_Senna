\chapter{Coleta de Dados}
\section{Determinação do Número de Sujeitos}

A determinação da amostra para este estudo foi realizada considerando as limitações impostas pela 
disponibilidade do equipamento \acrshort{MML} e as características físicas dos participantes. 
O estudo contou com a participação de cadetes do \acrlong{CBMSC}, sendo que, dentre os 16 
cadetes inicialmente considerados, foram selecionados apenas os 8 mais altos.

\tab A restrição imposta na seleção dos participantes deve-se ao fato de que a única amostra 
do uniforme \acrlong{MML} disponível para o estudo apresentava um tamanho incompatível com os 8 
cadetes de menor estatura. Assim, para garantir a adequada compatibilidade do uniforme e evitar 
variações indesejadas nos resultados decorrentes de ajustes inadequados dos fardamentos, 
optou-se por restringir a amostra àqueles cujas medidas corporais eram compatíveis com o uniforme 
fornecido pela coordenadoria de \acrlong{BTR}.

\tab Os demais uniformes utilizados no estudo eram de propriedade dos próprios cadetes, minimizando assim possíveis problemas relacionados à compatibilização do fardamento. Tal abordagem permitiu uma melhor padronização dos testes e assegurou que as diferenças observadas nos resultados estivessem predominantemente associadas às características dos uniformes, e não a problemas de ajuste das peças.

\section{Adaptações Realizadas}

Considerando as distinções intrínsecas entre os uniformes, objetos deste estudo, foram realizadas adaptações para minimizar diferenças nos comportamentos observados, conforme descrito a seguir:

\begin{enumerate}[label=\Roman*]
    \item \textbf{Padronização da Roupa Interna}: Foi definido o uniforme de \acrfull{EFM}, composto por calção de tactel e regata em material sintético, como a roupa interna para os três uniformes testados, a roupa de \acrshort{EFM} também foi considerada como linha de base, a fim de garantir que a principal variável seja o fardamento utilizado, além de facilitar a logística para a troca de uniforme;
    \item \textbf{Calçados}: Considerando que o foco do presente trabalho se trata dos fardamentos, 
    os calçados utilizados pelos participantes eram seus próprios tênis de corrida, com o objetivo de maximizar o conforto dos sujeitos e evitar lesões;
    \item \textbf{Diferenças Entre os Fardamentos}: 
    \begin{itemize}
        \item Os fardamentos analisados foram testados apenas com sua camada mais externa (calça e jaqueta);
        \item Devido ao fato dos \acrshort{EPI}s \acrshort{MM} e \acrshort{MML} não poderem ser utilizados com a manga recolhida, o uniforme 5ºA foi utilizado com sua manga abaixada.
    \end{itemize}
     
\end{enumerate}

\subsection{Comparação com Estudos Anteriores}

O critério de seleção dos participantes neste estudo difere do utilizado em pesquisas anteriores 
sobre testes de fardamentos operacionais em bombeiros, como o estudo conduzido por 
\textcite{orr2019impact}. No estudo de \textcite{orr2019impact}, foi avaliada a mobilidade de 
bombeiros utilizando diferentes variações de vestuário. A amostra de tal estudo foi composta por 
8 participantes, que testaram três variações de uniformes em comparação ao vestuário padrão da 
corporação \cite{orr2019impact}.

O presente estudo impôs uma restrição de altura devido à disponibilidade do equipamento, 
já na pesquisa de \textcite{orr2019impact} foi permitido uma variação mais ampla nas características 
antropométricas dos participantes. Isso pode ter implicações na generalização dos resultados, visto 
que diferenças na estatura e no ajuste do uniforme podem impactar diretamente na mobilidade e 
conforto dos usuários.

Enquanto a seleção de participantes no presente estudo foi guiada pela compatibilidade com 
um modelo específico de uniforme, \textcite{orr2019impact} testou a combinação de vestuários 
distintos e seus impactos sobre a mobilidade funcional. A abordagem adotada por eles permitiu 
avaliar a influência de camadas adicionais de vestuário sobre o desempenho motor dos bombeiros, 
um fator que não foi explorado na presente pesquisa.

Dessa forma, a metodologia de seleção adotada neste estudo proporciona um nível elevado de 
controle sobre a variável de ajuste do uniforme, apesar de reduzir a variabilidade da amostra.

A semelhança com o estudo de \textcite{orr2019impact}, consiste na utilização do \acrshort{FMS} e 
da análise subjetiva dos usuários como parâmetro de comparação entre os fardamentos. A pesquisa 
do presente trabalho diferencia-se do estudo de \textcite{orr2019impact}, principalmente, 
pela análise do comportamento térmico dos uniformes testados.


\section{\textit{Functional Movement Screen}}

Os testes do \acrlong{FMS} foram realizados em condições controladas, seguindo um protocolo 
padronizado para garantir a confiabilidade dos resultados. 

\subsection{Instrumentação}
Os instrumentos de medição utilizados no teste de \acrshort{FMS} foram construídos à partir de materiais 
básicos de construção civil, conforme é possível visualizar na Figura \ref{fig:materialFMS}, a instrumentação 
garante uma estrutura funcional para a realização dos testes. 
Os instrumentos foram utilizados para padronizar os critérios de avaliação e proporcionar 
uniformidade na aplicação dos testes. 

\figura{scale=0.25}{materialFMS.jpeg}{Instrumentos de medição}{fig:materialFMS}{}{do autor (2025)}

\subsection{Procedimento Experimental}
Os sujeitos foram avaliados em dias diferentes, utilizando um fardamento distinto a cada dia. 
Dessa forma, cada sujeito passou por todos os movimentos do \acrshort{FMS} sob cada uma das condições 
de fardamento, permitindo a comparação entre os diferentes uniformes.

A avaliação foi conduzida por um profissional de educação física capacitado, garantindo a 
aplicação correta dos protocolos do \acrshort{FMS}. A pontuação de cada movimento foi registrada 
em tabelas (Apêndice \ref{ap:fms}) de acordo com os critérios estabelecidos pelo \acrshort{FMS}, assegurando a 
consistência dos dados coletados.

\section{Comportamento Térmico}
A coleta de dados de comportamento térmico foi conduzida em um ambiente controlado, visando 
garantir a replicabilidade dos testes e a precisão dos resultados. O experimento foi 
estruturado da seguinte forma:
\subsection{Ambiente e Equipamentos Utilizados}

Os testes foram realizados em um ambiente fechado, climatizado, garantindo controle sobre fatores externos que poderiam influenciar nas medições de temperatura. O espaço era equipado com esteiras ergométricas para simular de forma consistente a atividade física dos participantes.

O equipamento de medição utilizado foi o \acrlong{DS} desenvolvido exclusivamente para este trabalho, cujos sensores foram estrategicamente posicionados, entre a roupa interna e o fardamento avaliado, para capturar a variação térmica em dois pontos principais do corpo, conforme diagrama da Figura \ref{fig:diagramaposicionamento}:

\begin{itemize}
    \item \textbf{Próximo à axila}: Região escolhida para monitoramento devido à sua proximidade com os grandes vasos sanguíneos, refletindo a variação térmica central do corpo.
    \item \textbf{Próximo à virilha}: Região selecionada por sua sensibilidade térmica e menor exposição ao fluxo de ar ambiente.
\end{itemize}    

\figura{scale=0.05}{posicionamentoSensores.jpg}{Diagrama de posicionamento dos sensores}{fig:diagramaposicionamento}{}{do autor (2025)}

O posicionamento estratégico visa evitar o atrito direto com a pele, minimizando possíveis interferências e garantindo a confiabilidade das medições, além de capturar dados em pontos-chave do corpo.

\subsection{Protocolo Experimental}

O protocolo do teste foi dividido em etapas sequenciais, conforme descrito a seguir:
\begin{enumerate}
    \item \textbf{Preparação}: O \acrlong{DS} era equipado ao militar, após isso, o participante vestia o fardamento a ser avaliado por cima do uniforme de \acrshort{EFM} e, após estar completamente equipado, o \acrshort{DS} iniciava a transmissão dos dados de temperatura.
    \item \textbf{Estabilização Inicial}: Um período de 2 minutos era aguardado antes do início da atividade física, permitindo que os sensores atingissem o equilíbrio térmico com a vestimenta do participante.
    \item \textbf{Atividade Física}: O sujeito realizava uma corrida de 5 minutos a uma velocidade constante de 12 km/h sobre a esteira ergométrica. Um estímulo físico constante e replicável para simular um esforço físico intenso.
    \item \textbf{Estabilização Final}: Após a corrida, um novo período de 2 minutos era aguardado para permitir que os sensores voltassem a um estado de equilíbrio térmico.
    \item \textbf{Finalização e Descanso}: Após a estabilização, o teste era encerrado, e o participante removia o uniforme avaliado, permanecendo apenas com o calção e a regata, em repouso, por pelo menos 10 minutos antes da realização de um novo teste com outro fardamento.
\end{enumerate}


\section{Formulário de Percepção Subjetiva}

A coleta de dados sobre as impressões dos participantes em relação aos diferentes uniformes testados foi realizada por meio do Formulário de Percepção Subjetiva. %Este instrumento foi aplicado imediatamente após a realização dos movimentos do \acrlong{FMS} para cada uma das variações de fardamento testadas. %O objetivo do formulário ser aplicado logo após os movimentos era capturar as percepções dos cadetes enquanto as sensações e limitações experimentadas ainda estavam vívidas, reduzindo possíveis vieses de memória, garantindo maior fidelidade nas respostas.

\tab Cada participante respondia ao formulário imediatamente após completar o protocolo do \acrshort{FMS} para um determinado uniforme. Essa abordagem sequencial minimizou possíveis interferências entre as percepções associadas a cada fardamento, permitindo uma melhor comparação entre os diferentes modelos testados.

A estratégia adotada assegurou que as avaliações fossem obtidas no momento mais oportuno, proporcionando um reflexo mais preciso da experiência dos participantes com cada fardamento. Além disso, ao manter a ordem padronizada na aplicação do \acrshort{FMS} e do formulário, foi possível reduzir possíveis efeitos de esquecimentos ou variabilidade na percepção dos cadetes.