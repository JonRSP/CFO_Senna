\chapter{Fardamentos Operacionais}

\section{Fardamento 5º A}

O fardamento utilizado por uma corporação militar tem papel fundamental na sua identidade visual, 
padronização e adequação às atividades operacionais. No \acrfull{CBMSC}, a implementação do 
fardamento 5ºA representou um marco na emancipação do corpo de bombeiros em relação à \acrfull{PMSC}, 
consolidando uma identidade própria em seu serviço à população catarinense.

\subsection{O Desenvolvimento do Fardamento 5ºA}

Segundo a entrevista realizada com o Coronel BM Diogo Bahia Losso (Apêndice \ref{ap:losso}), após 
a emancipação do \acrshort{CBMSC} em 2003, surgiu a necessidade da criação de um uniforme que 
representasse a nova identidade institucional. Até a emancipação, os bombeiros utilizavam o mesmo fardamento
dos policiais militares, na cor cáqui, diferenciando-se apenas pela cor da camiseta e do cinto (vermelhos) 
e pelo calçado utilizado, o borzeguim. A utilização da cor cáqui poderia comprometer a segurança dos
bombeiros militares em operações em locais de alto risco.

Para definir o novo fardamento, uma comissão foi estabelecida com o objetivo de criar um design 
exclusivo. A comissão, formada pelos coronéis Diogo Bahia Losso, Alexandre Corrêa Dutra e Marcos
 de Oliveira, enfrentou diversos desafios, entre eles destacam-se a escolha da cor e a seleção 
 do tecido ideal para as atividades desenvolvidas pelo \acrshort{CBMSC}.


\subsection{Evolução do Fardamento}

O uniforme operacional, em sua versão inicial, manteve o padrão tradicional adotado pela Polícia 
Militar, se diferenciando pela cor escolhida, o \textit{azul bandeirante}. No entanto, rapidamente 
se identificou a necessidade de adaptações para aprimorar sua resistência e funcionalidade. 
Inicialmente confeccionado em brim 100\% algodão, o fardamento apresentou um desbotamento precoce
 que comprometia sua aparência e padronização.

Para solucionar essa questão, foi introduzido o tecido \textit{terbrim}, composto por 67\% poliéster
 e 33\% algodão, com tecnologia \textit{ripstop}, proporcionando maior resistência ao desgaste 
 e durabilidade. Essa mudança exigiu um incentivo maior na utilização de 
 Equipamentos de Proteção Individuais (\acrshort{EPI}), visto que o novo tecido sintético apresentava riscos em situações de exposição 
 ao calor intenso de incêndios.

Além das mudanças nos materiais, o design do uniforme também evoluiu. O bolso faca foi removido, 
bolsos laterais foram adicionados, reforços em áreas estratégicas foram incorporados e foram utilizadas
técnicas com o tecido para aumentar a mobilidade dos militares. A gandola, que inicialmente 
era utilizada por dentro da calça, passou a ser usada por fora, conforme Figura \ref{fig:foto5a}, garantindo maior praticidade no 
serviço operacional e melhor apresentação pessoal (Apêndice \ref{ap:losso}).

\figura{scale=0.1}{5a.jpeg}{Fardamento 5ºA}{fig:foto5a}{}{do autor (2025)}

\subsection{Impacto e Desafios}

A introdução do fardamento 5ºA consolidou a identidade do \acrshort{CBMSC}, diferenciando a corporação das 
demais forças de segurança e tornando seus integrantes facilmente reconhecidos pela sociedade. 
Entretanto, desafios ainda permanecem, como a necessidade de adaptação às variações climáticas 
do estado de Santa Catarina. A ampla diversidade de temperaturas e condições ambientais exige 
que o uniforme ofereça conforto térmico e proteção adequada para todas as regiões do estado.

A evolução tecnológica dos tecidos e materiais apresenta-se como uma nova possibilidade
 para aprimorar o fardamento, o que permitiria maior eficiência e conforto para os bombeiros 
 militares (Apêndice \ref{ap:losso}). 


\section{EPI MultiMissão}

O \acrshort{EPI} \acrfull{MM} do \acrshort{CBMSC} foi regulamentado pela Resolução nº 16, de 7 de junho de 2023. Esse \acrlong{EPI} foi institucionalizado como 
vestimenta oficial para múltiplas operações, incluindo Resgate 
		Veicular, Combate a Incêndio Florestal, Salvamento em Altura, Corte de Árvores, 
		Intervenções em Áreas Deslizadas, Busca e Resgate em Estruturas Colapsadas, Busca Terrestre, Cinotecnia 
		e Atendimento Pré-Hospitalar \cite{res16}.

\subsection{Características Técnicas}

O \acrlong{MM}, conforme Figura \ref{fig:fotomm}, foi desenvolvido para substituir fardamentos que não atendiam plenamente 
às necessidades operacionais dos bombeiros militares. Seu diferencial, segundo o Anexo A 
da resolução 16/2023 \cite{res16}, está na combinação de tecidos de alta resistência térmica e 
mecânica, proporcionando mobilidade, proteção e conforto térmico.

O \acrshort{EPI} \acrshort{MM} é composto por blusão e calça, fabricados com meta-aramida, viscose FR, elastano, 
para-aramida e fibra antiestática. Essas características garantem resistência ao calor, 
inflamabilidade reduzida e proteção contra agentes cortantes. 
\\\\
Conforme Anexo A da resolução 16/2023 \cite{res16}, o design do uniforme também inclui:
\begin{itemize}
    \item Fechos em velcro e zíper termoplástico (Vislon FR);
    \item Faixas refletivas amarelas e prateadas, atendendo à norma EN 471;
    \item Reforços estruturais em para-aramida nos joelhos e cotovelos;
    \item Modelagem que permite ampla liberdade de movimento.
\end{itemize}


\figura{scale=0.1}{mm.jpeg}{EPI MultiMissão}{fig:fotomm}{}{do autor (2025)}


\subsection{Desenvolvimento e Implantação}

O desenvolvimento do \acrshort{EPI} \acrlong{MM} seguiu os princípios de \textit{Lean Innovation}, 
conforme descrito por \textcite{Melina2023}. A necessidade desse equipamento surgiu a partir 
da experiência prática de bombeiros militares catarinenses em competições de \acrfull{RVE}, 
onde foi identificado que o uso de \acrshort{EPI} de combate a incêndio estrutural 
comprometia a mobilidade e eficiência operacional.

O processo de inovação incluiu:
\begin{itemize}
    \item Testes em diferentes operações do \acrshort{CBMSC};
    \item Ajustes para garantir resistência térmica e mobilidade;
    \item Feedback contínuo de bombeiros militares envolvidos no desenvolvimento.
\end{itemize}

A aprovação do \acrlong{MM} resultou em sua regulamentação oficial pelo \acrshort{CBMSC} em 2023, 
sendo adotado por toda a corporação para operações terrestres de resgate.

\section{EPI MultiMissão Leve}

Com o objetivo de melhorar a segurança e a eficiência de suas equipes em campo, a Coordenadoria de \acrlong{BTR} propôs o desenvolvimento do \acrfull{MML}, um \acrshort{EPI} leve voltado para atividades específicas de \acrlong{BTR}, incluindo, também, as buscas cinotécnicas a partir da anuência da Coordenadoria de Cinotecnia. Essa proposta surge da necessidade de um uniforme mais adaptável às características das operações de longa duração, em terrenos acidentados e condições climáticas adversas.

O \acrshort{MML} busca solucionar limitações identificadas empiricamente nos uniformes operacionais atualmente utilizados pelo \acrshort{CBMSC}, proporcionando maior mobilidade, conforto térmico e resistência à abrasão. O desenvolvimento desse novo \acrshort{EPI} seguiu critérios técnicos baseados na experiência operacional dos bombeiros militares e em análises comparativas com equipamentos utilizados por outras corporações no Brasil.

\subsection{Desenvolvimento e Características do EPI Multimissão Leve}
    O \acrlong{MML}, conforme Figura \ref{fig:projetomml}, foi desenvolvido a partir da necessidade identificada pela Coordenadoria de \acrshort{BTR}, 
    e como parte interessada a Coordenadoria de Cinotecnia, visando solucionar limitações dos uniformes 
    operacionais preconizados para a atividade. O equipamento foi desenhado para oferecer maior 
    resistência à abrasão, proteção térmica adequada e flexibilidade durante a realização de 
    atividades físicas extenuantes, essenciais nas operações de busca \cite{relMML}. A Figura \ref{fig:fotomml} 
    mostra o protótipo utilizado para as avaliações executadas no presente estudo.

    Segundo o descritivo planejado pela coordenadoria de \acrlong{BTR} \cite{descritivoMML}, constante no anexo \ref{an:descritivo}, suas principais características consistem em:

\begin{itemize}
    \item Jaqueta 3x1 impermeável, com fleece interno acoplável, garantindo isolamento térmico e proteção contra intempéries;
    \item Calça reforçada com tecido Cordura 500 Denier nas regiões dos joelhos, nádegas e barras, aumentando a durabilidade e resistência ao desgaste;
    \item Fitas refletivas estrategicamente posicionadas para aumentar a visibilidade noturna;
    \item Identidade visual planejada conforme o \acrshort{EPI} \acrlong{MM} já institucionalizado.
\end{itemize}

\figura{scale=0.7}{MML_projeto.png}{Proposta de EPI MultiMissão Leve}{fig:projetomml}{}{CBMSC - Projeto Descritivo da Proposta de EPI Leve (2023) - \cite{descritivoMML} }
\figura{scale=0.1}{mml.jpeg}{Protótipo da Proposta de EPI MultiMissão Leve}{fig:fotomml}{}{do autor (2025)}

\subsection{Avaliação Técnica}

Para validar o desempenho do \acrlong{MML}, foram conduzidos testes em diferentes cenários operacionais pelas respectivas coordenadorias, incluindo operações de resgate aquático, salvamento em altura e corte de árvores. A avaliação foi realizada por bombeiros especializados nessas áreas, resultando em um relatório detalhado sobre os pontos positivos e as melhorias sugeridas \cite{relMML}.
\\
Os principais resultados apontaram que:
\begin{itemize}
    \item O conforto e mobilidade foram significativamente aprimorados, permitindo melhor desempenho dos bombeiros durante deslocamentos prolongados;
    \item A impermeabilização não atendeu completamente aos requisitos para operações em meio aquático, sendo necessária uma vedação aprimorada no capuz e nos zíperes;
    \item O sistema de reforço nos joelhos e cotovelos foi bem avaliado, mas sugeriu-se a troca do material para poliamida \textit{ripstop}, evitando rasgos e aumentando a vida útil do \acrshort{EPI}.
\end{itemize}

    

\subsection{Considerações}
O \acrlong{MML} se mostra como uma inovação essencial para aprimorar a eficácia das operações de \acrlong{BTR} no \acrshort{CBMSC}. Embora ajustes técnicos e diferentes testes ainda sejam necessários, os resultados até então obtidos indicam que o \acrshort{EPI} corresponde às demandas observadas pela coordenadoria de \acrshort{BTR}, oferecendo uma solução mais eficiente e segura para os bombeiros militares.