\chapter{Análise de Resultados}

\section{\textit{Functional Movement Screen}}

    Em relação aos dados obtidos nos testes análise estatística dos dados obtidos nos testes do \acrlong{FMS} realizados sob diferentes condições de fardamento. Foram aplicados testes estatísticos não paramétricos adequados para amostras pequenas e dados discretos, como o \textbf{teste de Friedman} e o \textbf{teste de Wilcoxon}, para identificar diferenças significativas entre as condições \cite{azevedo2018metodos}.

    \subsection{Teste de Friedman}
        O teste de Friedman é um teste estatístico não paramétrico utilizado para comparar três ou mais condições de medidas repetidas \cite{sheldon1996use}, 
        conforme Equação \ref{eq:friedman}, onde 
    $N$ é o número de sujeitos,
    $k$ é o número de tipos de fardamento comparados e
    $R_j$ é a soma dos \textit{ranks} para o fardamento $j$ no teste FMS. Ele verifica se há diferenças estatisticamente significativas entre 
        os fardamentos em relação ao desempenho nos testes de FMS.

\begin{equation}
\chi_F^2 = \left[ \frac{12}{Nk(k+1)} \sum_{j=1}^{k} R_j^2 \right] - 3N(k+1)
\label{eq:friedman}
\end{equation}

        A hipótese testada é:
        \begin{itemize}
            \item $H_0$: Não há diferenças entre os fardamentos.
            \item $H_1$: Pelo menos um fardamento apresenta diferenças estatisticamente significativas.
            \end{itemize}
        
            O $p$-valor calculado para cada teste representa a probabilidade de se obter um resultado 
            igual ou mais extremo do que o observado, assumindo que a hipótese nula ($H_0$) seja verdadeira. 
            O $p$-valor indica a evidência contra a hipótese nula: valores pequenos (geralmente 
            abaixo de 0,05) sugerem que a hipótese nula deve ser rejeitada, apontando para a existência de um 
            efeito significativo.

        A Tabela \ref{tab:friedman} apresenta os valores da estatística do teste de Friedman ($\chi^2_F$) e os respectivos \textit{p-valores}.

        \begin{table}[H]
            \centering
            \caption{Resultados do Teste de Friedman para os Movimentos do FMS}
            \label{tab:friedman}
            \begin{tabular}{lcc}
                \hline
                \textbf{Movimento} & \textbf{Estatística ($\chi^2_F$)} & \textbf{$p$-valor} \\
                \hline
                \textit{Active Straight Leg Raise} Direito  & 10.317 & 0.016  \\
                \textit{Active Straight Leg Raise} Esquerdo &  8.032 & 0.045  \\
                \textit{Deep Squat}                        &  4.000 & 0.261  \\
                \textit{Hurdle Step} Direito               & 13.047 & 0.004  \\
                \textit{Hurdle Step} Esquerdo              &  9.621 & 0.022  \\
                \hline
            \end{tabular}
            \end{table}

        Os resultados indicam que há diferenças estatisticamente significativas ($p < 0.05$) para os seguintes testes: \textbf{\textit{Active Straight Leg Raise} Direito e Esquerdo}, \textbf{\textit{Hurdle Step} Direito e Esquerdo}. Já os outros testes não apresentaram diferenças significativas entre os fardamentos.

    \subsection{Teste de Wilcoxon para Comparações Pareadas}
        Para determinar quais fardamentos diferem entre si, foi aplicado o teste de Wilcoxon para 
        comparações pareadas entre os fardamentos. O teste foi aplicado conforme a Equação \ref{eq:wilcoxon}, onde $d_i = x_i - y_i$ são as diferenças pareadas e $R(|d_i|)$ são os ranks das diferenças absolutas. 
        Esta ferramenta verifica diferenças entre duas condições específicas, sendo uma alternativa ao teste \textit{t de Student} pareado quando os dados não seguem distribuição normal \cite{azevedo2018metodos}.

    \begin{equation}
    W^- = \sum_{d_i < 0} R(|d_i|) + \frac{1}{2} \sum_{d_i = 0} R(|d_i|)
    \label{eq:wilcoxon}
    \end{equation}

        A Tabela \ref{tab:wilcoxon} apresenta os resultados do teste de Wilcoxon para os movimentos que mostraram diferenças significativas no teste de Friedman.

        \begin{table}[H]
            \centering
            \caption{Resultados do Teste de Wilcoxon para Comparação entre Fardamentos}
            \label{tab:wilcoxon}
            \begin{tabular}{lcc}
                \hline
                \textbf{Comparação} & \textbf{Estatística ($W$)} & \textbf{$p$-valor} \\
                \hline
                \textit{Active Straight Leg Raise} Direito (FMS 5ºA vs FMS BASE) & 175.50 & 0.71500 \\
                \textit{Active Straight Leg Raise} Direito (FMS 5ºA vs FMS MM) & 23.00 & 0.00006 \\
                \textit{Active Straight Leg Raise} Direito (FMS 5ºA vs FMS MML) & 40.00 & 0.05983 \\
                \textit{Active Straight Leg Raise} Direito (FMS BASE vs FMS MM) & 148.50 & 0.00086 \\
                \textit{Active Straight Leg Raise} Direito (FMS BASE vs FMS MML) & 96.00 & 0.04771 \\
                \textit{Active Straight Leg Raise} Direito (FMS MM vs FMS MML) & 104.00 & 0.02347 \\
                \textit{Active Straight Leg Raise} Esquerdo (FMS 5ºA vs FMS BASE) & 175.50 & 0.71500 \\
                \textit{Active Straight Leg Raise} Esquerdo (FMS 5ºA vs FMS MM) & 23.00 & 0.00006 \\
                \textit{Active Straight Leg Raise} Esquerdo (FMS 5ºA vs FMS MML) & 40.00 & 0.05983 \\
                \textit{Active Straight Leg Raise} Esquerdo (FMS BASE vs FMS MM) & 148.50 & 0.00086 \\
                \textit{Active Straight Leg Raise} Esquerdo (FMS BASE vs FMS MML) & 96.00 & 0.04771 \\
                \textit{Active Straight Leg Raise} Esquerdo (FMS MM vs FMS MML) & 104.00 & 0.02347 \\
                \textit{Hurdle Step} Direito (FMS 5ºA vs FMS BASE) & 175.50 & 0.71500 \\
                \textit{Hurdle Step} Direito (FMS 5ºA vs FMS MM) & 23.00 & 0.00006 \\
                \textit{Hurdle Step} Direito (FMS 5ºA vs FMS MML) & 40.00 & 0.05983 \\
                \textit{Hurdle Step} Direito (FMS BASE vs FMS MM) & 148.50 & 0.00086 \\
                \textit{Hurdle Step} Direito (FMS BASE vs FMS MML) & 96.00 & 0.04771 \\
                \textit{Hurdle Step} Direito (FMS MM vs FMS MML) & 104.00 & 0.02347 \\
                \textit{Hurdle Step} Esquerdo (FMS 5ºA vs FMS BASE) & 175.50 & 0.71500 \\
                \textit{Hurdle Step} Esquerdo (FMS 5ºA vs FMS MM) & 23.00 & 0.00006 \\
                \textit{Hurdle Step} Esquerdo (FMS 5ºA vs FMS MML) & 40.00 & 0.05983 \\
                \textit{Hurdle Step} Esquerdo (FMS BASE vs FMS MM) & 148.50 & 0.00086 \\
                \textit{Hurdle Step} Esquerdo (FMS BASE vs FMS MML) & 96.00 & 0.04771 \\
                \textit{Hurdle Step} Esquerdo (FMS MM vs FMS MML) & 104.00 & 0.02347 \\
                \hline
            \end{tabular}
            \end{table}

        Os resultados do teste de Wilcoxon mostram que o fardamento \acrlong{MM} impactou negativamente o desempenho no movimento Active Straight Leg Raise, pois apresentou diferenças significativas em relação à linha de base ($p = 0.00086$) e ao fardamento 5ºA ($p = 0.00006$). O fardamento \acrlong{MML} também influenciou esse movimento, mas com menor impacto ($p = 0.04772$).

    \subsection{Resumo}
        Os testes estatísticos indicam que os fardamentos podem afetar significativamente o desempenho 
        em alguns movimentos do \acrshort{FMS}, especialmente no \textit{\textbf{Active Straight Leg Raise}} e no \textit{\textbf{Hurdle 
        Step}}. Os principais achados são descritos a seguir:
        
        \begin{enumerate}[label=\Roman*.] % Usa números romanos maiúsculos
            \item O 5ºA apresentou os resultados mais próximos à linha de base, não atrapalhando a 
            movimentação.
            \item O \acrshort{EPI} \acrlong{MML} demonstrou mobilidade intermediária, com alguma
            restrição em certos padrões de movimento.
            \item O \acrshort{EPI} \acrlong{MM} apresentou o maior impacto na mobilidade, sendo 
            mais rígido em alguns pontos e com menor flexibilidade nas articulações.
            \end{enumerate}

        \tab Os resultados reforçam a importância de considerar a mobilidade e flexibilidade. 
        É possível perceber que os impactos negativos foram relacionados principalmente à 
        mobilidade de membros inferiores.

\section{Comportamento Térmico}
    A análise dos dados de temperatura tem como objetivo descrever o comportamento térmico 
    dos fardamentos testados.

    \subsection{Análise Gráfica}
        Os gráficos das figuras \ref{fig:temp5a}, \ref{fig:tempMM}, \ref{fig:tempMML} e \ref{fig:tempgeral} foram montados
        a partir dos dados coletados pelo sistema desenvolvido para este estudo. 

        O eixo X representa o tempo em segundos (s) do experimento. O eixo Y representa a variação de temperatura em graus Célsius (\degree C) para um 
        determinado sujeito utilizando um fardamento específico. A linha tracejada representa a 
        média de todas as leituras do mesmo uniforme. As linhas verticais representam as separações das etapas
        do experimento que sejam: 
        \begin{enumerate}[label=\Roman*.] % Usa números romanos maiúsculos
            \item Estabilização inicial (0s a 120s)
            \item Execução da atividade física (120s a 480s)
            \item Estabilização final (480s a 600s)
            \end{enumerate}
        

        Para avaliar a influência do sujeito na variação de temperatura em um determinado uniforme, foi 
        calculada a média dos valores de temperatura dos sujeitos em cada instante de tempo. Em seguida, 
        determinou-se o desvio padrão desses valores e, por fim, o \acrlong{CV} para cada momento da 
        coleta, conforme a equação \ref{eq:1}, por último foi calculada a média desses valores para cada 
        uniforme.

        \begin{equation} 
            \label{eq:1}CV=\frac{Desvio~Padr\tilde{a}o}{M\acute{e}dia}
            \end{equation}

        Além dessa análise, foi realizada uma regressão linear da média para cada etapa do experimento, 
        visando obter uma aproximação da taxa de variação da temperatura em cada fase do ensaio.

        \subsubsection{Fardamento 5ºA}
            Os dados coletados referentes ao fardamento 5ºA estão representados na Figura \ref{fig:temp5a}. 
            A tabela \ref{tab:est5a} presenta os valores máximos e mínimos de variação obtidos:

            \figura{scale=0.5}{grafico_temp_5a.png}{Gráfico da variação de temperatura por tempo do fardamento 5ºA}{fig:temp5a}{L}{do autor (2025)}


            \begin{table}[H]
            \centering
            \caption{Valores Máximos e Mínimos de Variação da Temperatura para o Fardamento 5ºA}
            \begin{tabular}{lccc}
            \hline
            Fardamento 5ºA & Variação máxima (\degree C) & Variação mínima (\degree C)\\ 
            \hline
            Virilha & 5.5 & 1.3 \\ 
            Axila & 5.0 & 2.1 \\ 
            \hline
            \end{tabular}
            
            \label{tab:est5a}
            \end{table}

            O \acrlong{CV} calculado para o fardamento 5ºA foi de \textbf{42.2\%} para a região da 
            virilha e \textbf{47.9\%} para a axila. As taxas de variação da média de temperatura 
            seguem a distribuição da tabela \ref{tab:taxa5a}:

            \begin{table}[h]
            \centering
            \caption{Taxa de Variação da Média de Temperatura para o Fardamento 5ºA}
            \begin{tabular}{lccc}
            \hline
            Fardamento 5ºA & I $(\degree C/min)$ & $II (\degree C/min)$ & $III (\degree C/min)$ \\ 
            \hline
            Virilha & 0.90 & 0.24 & 0.30 \\ 
            Axila & 0.62 & 0.13 & 0.50 \\ 
            \hline
            \end{tabular}
            
            \label{tab:taxa5a}
            \end{table}
        \subsubsection{EPI MultiMissão}
            Os dados coletados referentes ao \acrshort{EPI} \acrfull{MM} estão representados na Figura \ref{fig:tempMM}. 
            A tabela \ref{tab:estMM} presenta os valores máximos e mínimos de variação obtidos:


            \figura{scale=0.5}{grafico_temp_MM.png}{Gráfico da variação de temperatura por tempo do EPI \acrshort{MM}}{fig:tempMM}{L}{do autor (2025)}


            \begin{table}[H]
                \centering
                \caption{Valores Máximos e Mínimos de Variação da Temperatura para o EPI \acrshort{MM}}
                \begin{tabular}{lccc}
                \hline
                Fardamento MM & Variação máxima (\degree C) & Variação mínima (\degree C)\\ 
                \hline
                Virilha & 3.8 & 1.2 \\ 
                Axila & 4.5 & 0.8 \\ 
                \hline
                \end{tabular}
                
                \label{tab:estMM}
                \end{table}

            O \acrlong{CV} calculado para o \acrlong{MM} foi de \textbf{67.5\%} para ambas as regiões. 
            As taxas de variação da média de temperatura seguem a distribuição da tabela \ref{tab:taxaMM}:
            \begin{table}[h]
                \centering
                \caption{Taxa de Variação da Média de Temperatura para o EPI MM}
                \begin{tabular}{lccc}
                \hline
                Fardamento MM & $I (\degree C/min)$ & $II (\degree C/min)$ & $III (\degree C/min)$ \\ 
                \hline
                Virilha & 0.49 & 0.17 & 0.20 \\ 
                Axila & 0.34 & 0.26 & 0.15 \\ 
                \hline
                \end{tabular}
                
                \label{tab:taxaMM}
                \end{table}
        
        \subsubsection{Fardamento MultiMissão Leve}
            Os dados coletados referentes ao fardamento \acrfull{MML} estão representados na Figura \ref{fig:tempMML}. 
            A tabela \ref{tab:estMML} presenta os valores máximos e mínimos de variação obtidos:

            \figura{scale=0.5}{grafico_temp_MML.png}{Gráfico da variação de temperatura por tempo do EPI \acrshort{MML}}{fig:tempMML}{L}{do autor (2025)}

            \begin{table}[H]
                \centering
                \caption{Valores Máximos e Mínimos de Variação da Temperatura para o EPI \acrshort{MML}}
                    \begin{tabular}{lccc}
                    \hline
                    Fardamento MML & Variação máxima (\degree C) & Variação mínima (\degree C)\\ 
                    \hline
                    Virilha & 4.8 & 0.7 \\ 
                    Axila & 6.8 & 2.1 \\ 
                    \hline
                    \end{tabular}
                    
                    \label{tab:estMML}
                    \end{table}

            O \acrlong{CV} calculado para o \acrlong{MML} foi de \textbf{54.4\%} e \textbf{52.2\%} 
            para a região da virilha e da axila, respectivamente. 
            As taxas de variação da média de temperatura seguem a distribuição da tabela \ref{tab:taxaMML}:
            \begin{table}[h]
            \centering
            \caption{Taxa de variação da média de temperatura para o EPI MML}
            \begin{tabular}{lccc}
            \hline
            Fardamento MML & $I (\degree C/min)$ & $II (\degree C/min)$ & $III (\degree C/min)$ \\ 
            \hline
            Virilha & 0.59 & 0.12 & 0.39 \\ 
            Axila & 0.64 & 0.33 & 0.64 \\ 
            \hline
            \end{tabular}
            
            \label{tab:taxaMML}
            \end{table}

    \subsection{Comparação Entre Fardamentos}
        O gráfico da figura \ref{fig:tempgeral} apresenta uma comparação visual do comportamento médio dos 
        fardamentos analisados e a tabela \ref{tab:maxmedia} apresenta os valores da variação total de 
        temperatura da média para cada uniforme.

        \figura{scale=0.5}{grafico_temp_geral.png}{Gráfico da variação média de temperatura por tempo}{fig:tempgeral}{L}{do autor (2025)}

        \begin{table}[H]
            \centering
            \caption{Variação máxima da média de temperatura normalizada por fardamento}
            \begin{tabular}{lcc}
            \hline
            Fardamento & Variação da Média - Virilha & Variação da Média - Axila \\
            \hline
            5ºA & 3.9609 & 3.3281 \\ 
            MM & 2.4922 & 2.6797 \\ 
            MML & 3.0391 & 4.5078 \\ 
            \hline
            \end{tabular}
            
            \label{tab:maxmedia}
            \end{table}

        A partir dos dados apresentados, é possível verificar que o comportamento médio da temperatura 
        do fardamento \acrshort{MM}, no geral, aparenta ser o melhor. Os outros fardamentos 
        apresentam um comportamento diferente entre suas partes superiores e inferiores.

        Além disso, ao avaliar os valores de variação máxima e mínima de temperatura, foi possível estabelecer 
        uma classificação do comportamento térmico para os fardamentos em relação às diferentes partes do corpo:       

        \begin{itemize}
            \item \textbf{Partes inferiores:}
            \begin{enumerate}[label=\Roman*.] % Usa números romanos maiúsculos
            \item \acrlong{MM}
            \item \acrlong{MML}
            \item 5ºA
            \end{enumerate}

        
            \item \textbf{Partes superiores:}
            \begin{enumerate}[label=\Roman*.] % Usa números romanos maiúsculos
            \item \acrlong{MM}
            \item 5ºA
            \item \acrlong{MML}
            \end{enumerate}

        
            \end{itemize}
            
            O \acrlong{CV} médio para cada uniforme também foi analisado. O fardamento 5ºA apresentou 
            o menor coeficiente, indicando que este uniforme teve a menor variabilidade térmica entre 
            os usuários. Por outro lado, o MultiMissão apresentou o maior coeficiente, sugerindo uma 
            maior diferença na resposta térmica entre os participantes do experimento.

        Ao comparar os valores das taxas de variação de temperatura, observou-se que os 
        uniformes apresentam um comportamento semelhante. Durante a estabilização inicial, há um 
        taxa elevada de variação de temperatura, seguido de uma redução da taxa durante a execução do exercício 
        e, durante a estabilização final, há um novo aumento da taxa de variação de temperatura. 
        O fardamento \acrshort{MM} apresenta, no geral, a melhor taxa de variação de temperatura 
        dos três uniformes testados.

    \subsection{Resumo}
        A análise dos dados térmicos revelou diferenças significativas no comportamento dos 
        fardamentos testados. Os principais achados são descritos a seguir:
        
        \begin{enumerate}[label=\Roman*.] % Usa números romanos maiúsculos
            \item O \acrshort{EPI} \acrlong{MM} apresentou os melhores resultados médios em termos de 
            comportamento térmico, apesar de possuir o maior \acrlong{CV}, sugerindo maior dependência 
            sobre a fisiologia do militar.
            \item O 5ºA demonstrou comportamento térmico intermediário, apresentando uma variação de 
            temperatura elevada em sua parte inferior e menor variação em sua parte superior, mas por 
            ter apresentado o menor \acrlong{CV} de toda a amostra, se destaca positivamente por ter 
            um comportamento mais estável.
            \item O \acrshort{EPI} \acrlong{MML} apresentou o pior comportamento térmico considerando o 
            teste realizado, apresentando uma variação de temperatura menor em 
            sua parte inferior e maior variação em sua parte superior, além de 
            ter apresentado um \acrlong{CV} intermediário.
            \end{enumerate}   
        
        A redução das taxas de variação de temperatura durante a execução da atividade física, 
        pode ser explicada pelo fluxo de ar gerado durante a movimentação, resfriando o fardamento.

        O desempenho positivo da parte inferior do fardamento \acrshort{MM} pode ser explicado pela 
        diferença em seu design, a utilização de suspensórios ao invés do uso do cinto permite um maior 
        afastamento do tecido em relação ao corpo e, consequentemente, uma maior circulação de ar 
        durante a execução da atividade física.


\section{Formulário de Percepção Subjetiva}
    A análise do formulário de percepção subjetiva tem como objetivo compreender as percepções dos 
    participantes sobre o uso dos fardamentos avaliados. Foram coletadas informações sobre o nível de 
    incômodo gerado pelos diferentes fardamentos, os pontos específicos de desconforto e descrições 
    qualitativas das impressões dos usuários.

    \subsection{Análise Quantitativa:}
        
        Para avaliar o nível de incômodo gerado por cada fardamento, foi realizada uma análise 
        estatística descritiva, considerando médias, moda, desvio padrão e distribuição percentil. Os resultados 
        estão apresentados na Tabela \ref{tab:analise_fardamento}.
        \begin{table}[H]
                \centering
                \caption{Análise Quantitativa do Grau de Incômodo por Fardamento}
                \label{tab:analise_fardamento}
                \footnotesize
                \begin{tabular}{lrrrrrrrr}
                \toprule
                Fardamento &     Média &  Moda &  Desvio Padrão &  Mínimo &  1º Quartil &  Mediana &  3º Quartil &  Máximo \\
                \midrule
                5º A             &  1.888889 &     2 &       0.927961 &     1.0 &         1.0 &      2.0 &        2.00 &     4.0 \\
                MultiMissão      &  2.125000 &     2 &       0.640870 &     1.0 &         2.0 &      2.0 &        2.25 &     3.0 \\
                MultiMissão Leve &  1.375000 &     1 &       0.517549 &     1.0 &         1.0 &      1.0 &        2.00 &     2.0 \\
                \bottomrule
                \end{tabular}
                \end{table}
    

        Os resultados indicam que o fardamento MultiMissão Leve foi considerado o menos incômodo, 
        apresentando a menor média e menor desvio padrão, sugerindo maior conforto para os usuários. 
        O fardamento MultiMissão, embora tenha apresentado uma média próxima à do 5ºA, teve menor 
        variabilidade, indicando uma experiência mais consistente entre os participantes.
            

        %\figura{scale=0.4}{formulario_pontosFardamento.png}{Nuvem de palavras com os pontos de desconforto indicados por fardamento}
        %{fig:pontosFardamento}

 
    \subsection{ Análise Qualitativa:}
        Para aprofundar a análise, os participantes foram solicitados a indicar os pontos específicos 
        de desconforto e a descrever qualitativamente sua experiência com os fardamentos. Os pontos 
        de desconforto mais recorrentes foram representados no gráfico da figura \ref{fig:desconforto}. 
        A nuvem de palavras da figura \ref{fig:descricaoFardamento} destaca as sensações subjetivas 
        dos participantes. 
        
        \figura{scale=0.5}{grafico_desconforto.png}{Pontos de desconforto indicados por fardamento}
        {fig:desconforto}{}{do autor (2025)}

        \figura{scale=0.4}{formulario_descricaoFardamento.png}{Nuvem de palavras com as palavras-chave da descrição de desconforto indicados por fardamento}
        {fig:descricaoFardamento}{}{do autor (2025)}

        As descrições de percepção subjetiva reforçam que o peso e a flexibilidade do uniforme foram 
        fatores determinantes no conforto percebido pelos usuários. O fardamento \textbf{\acrlong{MML}} foi 
        frequentemente associado a maior liberdade de movimento.
    \subsection{Resumo}
    A análise subjetiva dos participantes permitiu identificar padrões de conforto e incômodo entre 
    os fardamentos testados. Os principais achados são:

    \begin{enumerate}[label=\Roman*.] % Usa números romanos maiúsculos
        \item O \acrshort{EPI} \acrlong{MML} apresentou a melhor avaliação subjetiva dos usuários, 
        sendo o fardamento com menos avaliações negativas e menos pontos de incômodo.
        \item O 5ºA demonstrou uma avaliação subjetiva intermediária, apresentando críticas 
        pontuais tanto na parte inferior quanto na parte superior do fardamento.
        \item O \acrshort{EPI} \acrlong{MM} apresentou a pior avaliação subjetiva 
        com a maior quantidade de relatos no campo descritivo e a maior incidência de críticas em 
        pontos específicos, em ambas as partes do fardamento. Destaca-se, em especial os joelhos 
        e os ombros.
        \end{enumerate}
