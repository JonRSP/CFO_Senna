\chapter{Metodologia}
\label{cap:metodologia}

A metodologia adotada neste trabalho fundamenta-se em três pilares principais: mobilidade funcional, 
comportamento térmico e percepção subjetiva conforme Figura \ref{fig:metodologia}. Essa estrutura foi inspirada no estudo conduzido por 
\textcite{orr2019impact}, que avaliou o impacto do vestuário na mobilidade de bombeiros. 

\figura{scale=0.4}{metodologia.png}{Diagrama dos pilares da metodologia}{fig:metodologia}{}{do autor (2025)}

Neste trabalho, a metodologia foi expandida para incluir a análise do comportamento térmico dos 
fardamentos, dada sua relevância operacional na realidade do CBMSC. De forma resumida, os pilares 
são explicados a seguir:

\section{Pilar 1: Avaliação da Mobilidade Funcional}

Para a avaliação da mobilidade funcional foi utilizado o protocolo \acrfull{FMS}. 
Este protocolo avalia sete padrões fundamentais de movimento: agachamento profundo, passo em barreira, 
avanço em linha, mobilidade de ombro, elevação ativa de perna, flexão com estabilidade de tronco e 
estabilidade rotacional.

Cada participante executou os testes com os três fardamentos analisados, permitindo comparações 
diretas entre os efeitos de cada vestimenta. A pontuação de cada movimento seguiu a escala proposta 
por \textcite{cook2006pre}, variando de 0 a 3, onde 0 indica dor ou incapacidade total de realizar o 
movimento e 3 representa execução perfeita sem compensações. A pontuação total varia de 0 a 21 pontos.

\section{Pilar 2: Análise do Comportamento Térmico}

Para analisar o comportamento térmico dos fardamentos foi desenvolvido um sistema de coleta de dados 
composto por sensores de temperatura distribuídos em pontos estratégicos do corpo. O sistema foi 
previamente validado em ambiente laboratorial.

Durante o experimento, os sensores registraram continuamente a temperatura da superfície 
interna do uniforme enquanto os participantes realizavam atividades físicas padronizadas. As variáveis 
analisadas incluíram a temperatura média, os valores máximos e mínimos e a taxa de variação ao longo 
do tempo.

\section{Pilar 3: Percepção Subjetiva}

O terceiro pilar metodológico consistiu na aplicação de um formulário estruturado que avaliou a 
percepção subjetiva dos participantes após o uso de cada fardamento. O instrumento investigou o grau 
de desconforto percebido, os locais anatômicos onde ele ocorreu e o tipo de sensação experimentada.

A escala utilizada variou de 1 a 5, sendo 1 equivalente a nenhum incômodo e 5 equivalente a máximo 
incômodo percebido. As respostas foram categorizadas e analisadas para identificar padrões de 
desconforto associados a cada tipo de uniforme.

\section{Integração dos Dados}

Os dados obtidos nos três pilares foram analisados de forma integrada. A triangulação metodológica 
permite comparar os fardamentos com base nas evidências coletadas, fornecendo subsídios 
para recomendar aquele que oferece melhor desempenho geral, e no caso avaliado neste trabalho, 
para a atividade de busca terrestre.
