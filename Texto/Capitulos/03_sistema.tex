\chapter{Sistema para Captura do Comportamento Térmico}
A análise térmica de fardamentos operacionais é um fator essencial para avaliar o conforto e 
a segurança dos profissionais em ambientes desafiadores. Considerando a importância desse aspecto, 
este estudo desenvolveu um sistema para captação de dados de temperatura que permite monitorar 
e comparar o desempenho térmico de diferentes uniformes. A metodologia adotada envolveu a 
utilização de sensores de temperatura acoplados a dispositivos eletrônicos, garantindo a coleta 
sistemática e precisa das variações térmicas durante a atividade física dos participantes.


\section{Arquitetura do Sistema}

A norma ISO/IEC/IEEE 42010:2011 define "arquitetura"~ como os conceitos ou propriedades fundamentais de um sistema em seu ambiente e os princípios de seu design e evolução. A referência descreve que essas propriedades ou conceitos fundamentais devem estar incorporados nos elementos do sistema e em suas relações \cite{ISO42010}. Essa definição é orientada para enfatizar as características essenciais e as interações entre os componentes do sistema.

\subsection{Internet das Coisas}

	A \acrlong{IoT}, do inglês \textit{Internet of things} (\acrshort{IoT}), consiste na interconexão de dispositivos sensitivos e atuadores com a finalidade de atingir um
	 objetivo em comum \cite{giusto2010internet} e tem como propósito primário permitir que humanos e máquinas compreendam melhor o ambiente que os envolve,
	  usando as informações geradas por diversos dispositivos sensitivos, modificando a forma com que os usuários lidam com as tarefas do cotidiano \cite{atzori2010internet}.
	
	A \acrshort{IoT} é um paradigma tecnológico planejado para ser uma rede global de máquinas e dispositivos capazes de interagir entre si e com o ambiente ao
	 seu redor. A \acrshort{IoT} é reconhecida como uma das áreas mais importantes da tecnologia do futuro por enfatizar a interoperabilidade entre objetos e pessoas
	  e pelo fato de poder ser implementada em diversos casos de uso, como por exemplo, automação predial, controle de processos produtivos e transporte inteligente \cite{gubbi2013internet}.
	
	  No presente trabalho, foram utilizados os conceitos de \acrlong{IoT} para o desenvolvimento da arquitetura para coleta dos dados do comportamento térmico
	   dos fardamentos operacionais.

\subsection{Dispositivo Sensitivo}

	O \acrfull{DS} desenvolvido para este trabalho baseia-se no microcontrolador ESP32, reconhecido por sua alta capacidade de processamento e conectividade
	 integrada Wi-Fi e Bluetooth. O dispositivo é projetado para operar como um coletor de dados, utilizando inicialmente dois sensores de temperatura modelo DS18B20,
	  que oferecem alta precisão e confiabilidade na medição. O \acrshort{DS} apresenta flexibilidade para expansão, permitindo a adição de sensores adicionais conforme a
	   necessidade, garantindo adaptabilidade ao sistema. O esquema de montagem segue o diagrama da Figura \ref{fig:diagramaDS}, tendo sido construídos para
	    este trabalho duas unidades conforme a figura.

	\figura{scale=0.4}{esquema_DS.jpg}{Diagrama de montagem dos dispositivos sensitivos}{fig:diagramaDS}

\begin{centering}
	[Adicionar Foto dos sensores]
\end{centering}

	\tab A comunicação com o servidor ocorre via protocolo \acrfull{MQTT}, uma solução leve e eficiente, ideal para aplicações de
	 \acrlong{IoT}. Este protocolo assegura uma troca de mensagens ágil e segura entre o dispositivo e a infraestrutura central, 
	 possibilitando o envio de dados em tempo real e a manutenção de um sistema responsivo para monitoramento e análise. O conjunto do ESP32 e
	  os sensores DS18B20, integrado por meio do \acrshort{MQTT}, configura o \acrshort{DS} como uma solução robusta para o teste proposto.
	
\section{Sistema para Coleta de Dados}

	O sistema desenvolvido para coleta de dados de temperatura utiliza princípios da \acrlong{IoT}. 
	O \acrlong{DS} se conecta a uma rede Wi-Fi disponível e uma vez conectado, esse dispositivo envia 
	mensagens via protocolo \acrshort{MQTT} para um servidor central. Esse servidor, por sua vez, executa um programa responsável por armazenar os conteúdos das mensagens em um banco de dados relacional, garantindo a organização e acessibilidade dos dados coletados. O diagrama da Figura \ref{fig:diagramaSis} representa uma visão simplificada da arquitetura do sistema de coleta de dados.

    \figura{scale=0.4}{sistema_diagrama.png}{Diagrama simplificado da arquitetura do sistema de coleta de dados de temperatura}{fig:diagramaSis}

\tab O planejamento do sistema visou eliminar restrições geográficas, permitindo a realização de testes em diferentes locais. Enquanto houver conexão com a internet, os testes podem ser conduzidos simultaneamente com diversos dispositivos sensitivos em qualquer região. Essa flexibilidade possibilita que futuros estudos baseados neste trabalho possam ser realizados em diferentes Batalhões do Corpo de Bombeiros Militar, ampliando a abrangência da pesquisa e favorecendo a padronização de dados para análises comparativas.

	
	
\section{Validação dos Dispositivos Sensitivos}
    Para assegurar a confiabilidade dos dados coletados, os sensores foram submetidos a um processo 
	rigoroso de validação em laboratório, sendo comparados com equipamentos de referência industrial. 
	Durante os testes, foram aplicadas variações controladas de temperatura, permitindo analisar a 
	precisão das medições e identificar possíveis desvios.

	A validação dos dados coletados pelos sensores desenvolvidos foi realizada no Centro de Inovação 
	e Pesquisa do \acrshort{CBMSC}, os dispositivos sensitivos
	 construídos foram posicionados junto a sensores industriais, pertencentes ao laboratório, os 
	 quais foram utilizados como referência. Esse procedimento, demonstrado pela Figura 
	 \ref{fig:experimento} teve como objetivo verificar a precisão, confiabilidade e correlação das 
	 medições obtidas pelos equipamentos desenvolvidos em relação aos sensores de padrão industrial.
	
	 \figura{scale=0.37}{validacao_1.jpg}{Montagem do experimento em laboratório}{fig:experimento}
	
	Inicialmente, foi aguardado um período de estabilização térmica do sistema, garantindo que 
	os sensores estivessem em condições homogêneas de medição. Após essa fase, foram 
	realizadas variações controladas de temperatura, utilizando água aquecida para elevar a 
	temperatura do ambiente e gelo para provocar reduções térmicas. Durante esse processo, as 
	medições de temperatura dos dispositivos construídos foram registradas e comparadas com 
	os valores obtidos pelos sensores industriais conforme a Figura \ref{fig:comparacao}.
	
    
    \figura{scale=0.5}{validacao_comparacao.png}{Comparação entre os valores obtidos pelos dispositivos construídos e sensores de referência}{fig:comparacao}


	\subsection{Analíses Quantitativas}
	
	A validação foi conduzida por meio de análises estatísticas, comparando as medições dos sensores construídos com os sensores industriais de referência. Os principais parâmetros analisados foram:


	\begin{itemize}
		\item  Erro Médio Absoluto (MAE - \textit{Mean Absolute Error})
		\end{itemize}
		\tab O \acrshort{MAE} foi utilizado para medir a diferença absoluta média entre os valores dos sensores desenvolvidos e os valores de referência. Quanto menor o \acrshort{MAE}, maior a precisão do sensor.
			\begin{itemize}
				\item \acrshort{DS} 1: \acrshort{MAE} para temperatura (valor A) = 0,998°C e (valor V) = 1,327°C.
				\item \acrshort{DS} 2: \acrshort{MAE} para temperatura (valor A) = 1,109°C e (valor V) = 1,204°C.
			\end{itemize}

		Os valores de erro absoluto indicam uma boa proximidade das medições em relação às referências.

	\begin{itemize}
		\item Erro Médio Percentual Absoluto (MAPE - \textit{Mean Absolute Percentage Error})
		\end{itemize}
		\tab O \acrshort{MAPE} avalia o erro relativo das medições em relação à referência, permitindo uma comparação percentual.
		\begin{itemize}
			\item \acrshort{DS} 1: \acrshort{MAPE} para temperatura (valor A) = 3,55\% e (valor V) = 4,35\%.
			\item \acrshort{DS} 2: \acrshort{MAPE} para temperatura (valor A) = 3,76\% e (valor V) = 3,74\%.
		\end{itemize}

		Ambos os dispositivos apresentam erros percentuais abaixo de 5\%, o que indica que 
		os equipamentos construídos apresentam precisão aceitável para o objetivo do trabalho.
	
	\begin{itemize}	
		\item Coeficiente de Correlação de Pearson
		\end{itemize}

		O coeficiente de correlação de Pearson foi utilizado para medir a relação linear 
		entre as temperaturas registradas pelos dispositivos desenvolvidos e os sensores de referência.
		
		\begin{itemize}
			\item \acrshort{DS} 1: Correlação de valores de temperatura (valor A) = 0,982 e (valor V) = 0,976.
			\item \acrshort{DS} 2: Correlação de valores de temperatura (valor A) = 0,980 e (valor V) = 0,975.
		\end{itemize}
		
		Ambos os \acrshort{DS} apresentam valores próximos de 1,0 o que indica correlação 
		extremamente forte, demonstrando que os equipamentos seguem a mesma tendência dos sensores de 
		referência.

	\begin{itemize}	
		\item Distribuição de Erros
	\end{itemize}
	
		A distribuição dos erros foi analisada para verificar se os desvios apresentavam padrões 
		sistemáticos ou aleatórios. Os histogramas, mostrados na Figura \ref{fig:erro} 
		demonstraram que os erros estavam concentrados próximos de zero, confirmando a baixa variabilidade e a alta precisão dos dispositivos construídos.

        \figura{scale=0.5}{validacao_erro.png}{Histogramas de distribuição de erros}{fig:erro}{L}

	

\subsection{Conclusões da Validação dos Dispositivos Sensitivos}
	Os resultados obtidos demonstram que os dispositivos desenvolvidos possuem alta precisão e confiabilidade, apresentando erros absolutos pequenos e correlações extremamente fortes com os sensores de referência. 
    
    O erro percentual médio abaixo de 5\% indica que os Dispositivos Sensitivos fornecem medições coerentes com padrões de referência, demonstrando a viabilidade do sistema proposto para o estudo.