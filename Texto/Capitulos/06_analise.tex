\chapter{ANÁLISE DE RESULTADOS}

\section{\acrshort{FMS}}

Em relação aos dados obtidos nos testes análise estatística dos dados obtidos nos testes do \acrlong{FMS} realizados sob diferentes condições de fardamento. Foram aplicados testes estatísticos não paramétricos adequados para amostras pequenas e dados discretos, como o \textbf{teste de Friedman} e o \textbf{teste de Wilcoxon}, para identificar diferenças significativas entre as condições.

\subsection{Teste de Friedman}
O teste de Friedman é um teste estatístico não paramétrico utilizado para comparar três ou mais condições de medidas repetidas \cite{sheldon1996use}. Ele verifica se há diferenças estatisticamente significativas entre os fardamentos em relação ao desempenho nos testes de FMS.

A hipótese testada é:
\begin{itemize}
    \item $H_0$: Não há diferenças entre os fardamentos.
    \item $H_1$: Pelo menos um fardamento apresenta diferenças estatisticamente significativas.
\end{itemize}

A Tabela \ref{tab:friedman} apresenta os valores da estatística do teste de Friedman ($\chi^2_F$) e os respectivos \textit{p-valores}.

\begin{table}[H]
    \centering
    \caption{Resultados do Teste de Friedman para os Movimentos do FMS}
    \label{tab:friedman}
    \begin{tabular}{lcc}
        \hline
        \textbf{Movimento} & \textbf{Estatística ($\chi^2_F$)} & \textbf{$p$-valor} \\
        \hline
        Active Straight Leg Raise Direito  & 10.317 & 0.016  \\
        Active Straight Leg Raise Esquerdo &  8.032 & 0.045  \\
        Deep Squat                        &  4.000 & 0.261  \\
        Hurdle Step Direito               & 13.047 & 0.004  \\
        Hurdle Step Esquerdo              &  9.621 & 0.022  \\
        \hline
    \end{tabular}
\end{table}

Os resultados indicam que há diferenças estatisticamente significativas ($p < 0.05$) para os seguintes testes: \textbf{Active Straight Leg Raise Direito e Esquerdo}, \textbf{Hurdle Step Direito e Esquerdo}. Já os outros testes não apresentaram diferenças significativas entre os fardamentos.

\subsection{Teste de Wilcoxon para Comparações Pareadas}
Para determinar quais fardamentos diferem entre si, foi aplicado o teste de Wilcoxon para comparações pareadas entre os fardamentos. Este teste verifica diferenças entre duas condições específicas, sendo uma alternativa ao teste \textit{t de Student} pareado quando os dados não seguem distribuição normal \cite{hilton1996appropriateness}.

A Tabela \ref{tab:wilcoxon} apresenta os resultados para os movimentos que mostraram diferenças significativas no teste de Friedman.

\begin{table}[H]
    \centering
    \caption{Resultados do Teste de Wilcoxon para Comparação entre Fardamentos}
    \label{tab:wilcoxon}
    \begin{tabular}{lcc}
        \hline
        \textbf{Comparação} & \textbf{Estatística ($W$)} & \textbf{$p$-valor} \\
        \hline
        Active Straight Leg Raise Direito (FMS 5ºA vs FMS BASE) & 175.50 & 0.71500 \\
        Active Straight Leg Raise Direito (FMS 5ºA vs FMS MM) & 23.00 & 0.00006 \\
        Active Straight Leg Raise Direito (FMS 5ºA vs FMS MML) & 40.00 & 0.05983 \\
        Active Straight Leg Raise Direito (FMS BASE vs FMS MM) & 148.50 & 0.00086 \\
        Active Straight Leg Raise Direito (FMS BASE vs FMS MML) & 96.00 & 0.04771 \\
        Active Straight Leg Raise Direito (FMS MM vs FMS MML) & 104.00 & 0.02347 \\
        Active Straight Leg Raise Esquerdo (FMS 5ºA vs FMS BASE) & 175.50 & 0.71500 \\
        Active Straight Leg Raise Esquerdo (FMS 5ºA vs FMS MM) & 23.00 & 0.00006 \\
        Active Straight Leg Raise Esquerdo (FMS 5ºA vs FMS MML) & 40.00 & 0.05983 \\
        Active Straight Leg Raise Esquerdo (FMS BASE vs FMS MM) & 148.50 & 0.00086 \\
        Active Straight Leg Raise Esquerdo (FMS BASE vs FMS MML) & 96.00 & 0.04771 \\
        Active Straight Leg Raise Esquerdo (FMS MM vs FMS MML) & 104.00 & 0.02347 \\
        Hurdle Step Direito (FMS 5ºA vs FMS BASE) & 175.50 & 0.71500 \\
        Hurdle Step Direito (FMS 5ºA vs FMS MM) & 23.00 & 0.00006 \\
        Hurdle Step Direito (FMS 5ºA vs FMS MML) & 40.00 & 0.05983 \\
        Hurdle Step Direito (FMS BASE vs FMS MM) & 148.50 & 0.00086 \\
        Hurdle Step Direito (FMS BASE vs FMS MML) & 96.00 & 0.04771 \\
        Hurdle Step Direito (FMS MM vs FMS MML) & 104.00 & 0.02347 \\
        Hurdle Step Esquerdo (FMS 5ºA vs FMS BASE) & 175.50 & 0.71500 \\
        Hurdle Step Esquerdo (FMS 5ºA vs FMS MM) & 23.00 & 0.00006 \\
        Hurdle Step Esquerdo (FMS 5ºA vs FMS MML) & 40.00 & 0.05983 \\
        Hurdle Step Esquerdo (FMS BASE vs FMS MM) & 148.50 & 0.00086 \\
        Hurdle Step Esquerdo (FMS BASE vs FMS MML) & 96.00 & 0.04771 \\
        Hurdle Step Esquerdo (FMS MM vs FMS MML) & 104.00 & 0.02347 \\
        \hline
    \end{tabular}
\end{table}

Os resultados do teste de Wilcoxon mostram que o fardamento \acrlong{MM} impactou negativamente o desempenho no movimento Active Straight Leg Raise, pois apresentou diferenças significativas em relação à linha de base ($p = 0.00086$) e ao fardamento 5ºA ($p = 0.00006$). O fardamento \acrlong{MML} também influenciou esse movimento, mas com menor impacto ($p = 0.04772$).

\subsection{Conclusões}
Os testes estatísticos indicam que os fardamentos afetam significativamente o desempenho em alguns movimentos do \acrshort{FMS}, especialmente no Active Straight Leg Raise e no Hurdle Step. O fardamento \acrlong{MM} demonstrou maior impacto negativo, enquanto o uniforme 5ºA apresentou desempenho semelhante à linha de base.

\tab Os resultados reforçam a importância de considerar a mobilidade e flexibilidade. É possível perceber que os impactos negativos foram relacionados principalmente à mobilidade de membros inferiores.

\section{Dados de Temperatura}


A análise dos dados de temperatura tem como objetivo descrever o comportamento térmico dos fardamentos analisados.

\subsection{Análise Gráfica}
Os gráficos das figuras \ref{fig:temp5a}, \ref{fig:tempMM}, \ref{fig:tempMML} e \ref{fig:tempgeral} foram montados
a partir dos dados coletados pelo sistema desenvolvido para este trabalho. 

O eixo X representa o tempo em segundos (s) do experimento, o eixo Y representa a variação de temperatura em graus Célsius (\degree C) para um 
determinado sujeito utilizando um fardamento específico. A linha tracejada representa a 
média de todas as leituras do mesmo uniforme. As linhas verticais representam as separações das etapas
do experimento que sejam: 
\begin{enumerate}[label=\Roman*.] % Usa números romanos maiúsculos
    \item Estabilização inicial (0s a 120s)
    \item Execução da atividade física (120s a 480s)
    \item Estabilização final (480s a 600s)
\end{enumerate}
 

A fim de se avaliar a influência do sujeito na variação de temperatura em um determinado
uniforme, para cada fardamento, individualmente, foi calculada a média dos valores de temperatura 
dos sujeitos em cada instante de tempo, em seguida, o desvio padrão dos valores de temperatura 
nesse instante, após isso, foi calculado o \acrlong{CV} para cada 
momento da coleta conforme a equação \ref{eq:1}, por último foi calculada a média desses valores para cada uniforme.


\begin{equation} \label{eq:1}CV=\frac{Desvio~Padr\tilde{a}o}{M\acute{e}dia}\end{equation}

Além dessa análise, também foi realizada uma regressão linear para cada etapa do 
experimento com o objetivo de se ter uma aproximação da taxa de variação da temperatura
 a cada fase do ensaio.

\subsubsection{Fardamento 5ºA}
Para os dados coletados no experimento referentes ao fardamento 5ºA, foi gerado o gráfico da figura \ref{fig:temp5a} a seguir:

    \figura{scale=0.5}{grafico_temp_5a.png}{Gráfico da variação de temperatura por tempo do fardamento 5ºA}{fig:temp5a}{L}

A tabela \ref{tab:est5a} apresenta os valores máximos e mínimos de variação obtidos com este uniforme.

\begin{table}[H]
    \centering
    \begin{tabular}{lccc}
    \hline
    Fardamento 5ºA & Variação máxima (\degree C) & Variação mínima (\degree C)\\ 
    \hline
    Virilha & 5.5 & 1.3 \\ 
    Axila & 5.0 & 2.1 \\ 
    \hline
    \end{tabular}
    \caption{Valores máximos e mínimos de variação da temperatura para o fardamento 5ºA}
    \label{tab:est5a}
    \end{table}

O \acrlong{CV} calculado para o 5ºA foi de \textbf{42.2\%} para a região da virilha e \textbf{47.9\%} para a região da axila.

Já suas taxas de variação de temperatura seguem a distribuição da tabela \ref{tab:taxa5a}:

\begin{table}[h]
    \centering
    \begin{tabular}{lccc}
    \hline
    Fardamento 5ºA & I $(\degree C/min)$ & $II (\degree C/min)$ & $III (\degree C/min)$ \\ 
    \hline
    Virilha & 0.90 & 0.24 & 0.30 \\ 
    Axila & 0.62 & 0.13 & 0.50 \\ 
    \hline
    \end{tabular}
    \caption{Taxa de variação da temperatura para o fardamento 5ºA}
    \label{tab:taxa5a}
    \end{table}
\subsubsection{Fardamento \acrlong{MM}}
Para os dados coletados no experimento referentes ao fardamento \acrlong{MM}, foi gerado o gráfico da figura \ref{fig:tempMM} a seguir:

    \figura{scale=0.5}{grafico_temp_MM.png}{Gráfico da variação de temperatura por tempo do fardamento \acrshort{MM}}{fig:tempMM}{L}

A tabela \ref{tab:estMM} apresenta os valores máximos e mínimos de variação obtidos com este uniforme.

    \begin{table}[H]
        \centering
        \begin{tabular}{lccc}
        \hline
        Fardamento MM & Variação máxima (\degree C) & Variação mínima (\degree C)\\ 
        \hline
        Virilha & 3.8 & 1.2 \\ 
        Axila & 4.5 & 0.8 \\ 
        \hline
        \end{tabular}
        \caption{Valores máximos e mínimos de variação da temperatura para o fardamento \acrshort{MM}}
        \label{tab:estMM}
        \end{table}

O \acrlong{CV} calculado para o \acrlong{MM} foi de \textbf{67.5\%} para ambas as regiões.

Já suas taxas de variação de temperatura seguem a distribuição da tabela \ref{tab:taxaMM}:
\begin{table}[h]
        \centering
        \begin{tabular}{lccc}
        \hline
        Fardamento MM & $I (\degree C/min)$ & $II (\degree C/min)$ & $III (\degree C/min)$ \\ 
        \hline
        Virilha & 0.49 & 0.17 & 0.20 \\ 
        Axila & 0.34 & 0.26 & 0.15 \\ 
        \hline
        \end{tabular}
        \caption{Taxa de variação da temperatura para o fardamento MM}
        \label{tab:taxaMM}
        \end{table}
        
\subsubsection{Fardamento \acrlong{MML}}
Para os dados coletados no experimento referentes ao fardamento \acrlong{MML}, foi gerado o gráfico da figura \ref{fig:tempMML} a seguir:

    \figura{scale=0.5}{grafico_temp_MML.png}{Gráfico da variação de temperatura por tempo do fardamento \acrshort{MML}}{fig:tempMML}{L}

A tabela \ref{tab:estMML} apresenta os valores máximos e mínimos de variação obtidos com este uniforme.

    \begin{table}[H]
        \centering
        \begin{tabular}{lccc}
        \hline
        Fardamento MML & Variação máxima (\degree C) & Variação mínima (\degree C)\\ 
        \hline
        Virilha & 4.8 & 0.7 \\ 
        Axila & 6.8 & 2.1 \\ 
        \hline
        \end{tabular}
        \caption{Valores máximos e mínimos de variação da temperatura para o fardamento \acrshort{MML}}
        \label{tab:estMML}
        \end{table}

    O \acrlong{CV} calculado para o \acrlong{MML} foi de \textbf{54.4\%} e \textbf{52.2\%} para a região da virilha e da axila, respectivamente.

   Já suas taxas de variação de temperatura seguem a distribuição da tabela \ref{tab:taxaMML}:
\begin{table}[h]
    \centering
    \begin{tabular}{lccc}
    \hline
    Fardamento MML & $I (\degree C/min)$ & $II (\degree C/min)$ & $III (\degree C/min)$ \\ 
    \hline
    Virilha & 0.59 & 0.12 & 0.39 \\ 
    Axila & 0.64 & 0.33 & 0.64 \\ 
    \hline
    \end{tabular}
    \caption{Taxa de variação da temperatura para o fardamento MML}
    \label{tab:taxaMML}
    \end{table}

\subsection{Comparação Entre Fardamentos}
Foi desenvolvido o gráfico da figura \ref{fig:tempgeral} com o objetivo de se comparar visualmente o
 comportamento médio dos fardamentos.

\figura{scale=0.5}{grafico_temp_geral.png}{Gráfico da variação média de temperatura por tempo}{fig:tempgeral}{L}

Ao se comparar o \acrlong{CV} médio de cada uniforme, é possível verificar que o fardamento 5ºA apresenta 
o menor coeficente, o que indica que dentre os três fardamentos testados, este é o que apresenta a menor 
variabilidade em relação ao usuário e o \acrlong{MM} é o que apresenta uma maior variabilidade em relação 
o usuário.

Comparando os valores das variações máximas e mínimas de cada fardamento, bem como o valor máximo de variação 
do comportamento médio, é possível perceber que o 

Comparando os valores das taxas de variação de temperatura, é possível perceber que os uniformes 
apresentam um comportamento semelhante, durante a estabilização inicial, há uma taxa elevada de 
variação de temperatura, durante a execução do exercício, a taxa reduz significativamente e durante a 
estabilização final, há um novo aumento da taxa de variação de temperatura.

O comportamento do fardamento \acrshort{MM} na região da virilha seguiu o padrão visualizado, mas de uma forma 
a se destacar em relação aos outros,  


\subsection{Conclusões}
A partir dos valores calculados de \acrlong{CV}, é possível notar que o fardamento 5ºA apresenta 
o menor coeficiente, mostrando que 

\section{Formulário de Percepção Subjetiva}
A análise do formulário de percepção subjetiva tem como objetivo compreender as percepções dos participantes sobre o uso dos fardamentos avaliados. 

    \subsection{Análise Quantitativa:}
    \begin{itemize}
    \item A questão sobre a percepção do incômodo gerado pelo fardamento foi analisadas por meio de estatísticas descritivas, como médias e distribuições percentuais contidas na Tabela \ref{tab:analise_fardamento}.
    \begin{table}[H]
\centering
\caption{Análise Quantitativa do Grau de Incômodo por Fardamento}
\label{tab:analise_fardamento}
\footnotesize
\begin{tabular}{lrrrrrrrr}
\toprule
{} &     Média &  Moda &  Desvio Padrão &  Mínimo &  1º Quartil &  Mediana &  3º Quartil &  Máximo \\
Fardamento       &           &       &                &         &             &          &             &         \\
\midrule
5º A             &  1.888889 &     2 &       0.927961 &     1.0 &         1.0 &      2.0 &        2.00 &     4.0 \\
MultiMissão      &  2.125000 &     2 &       0.640870 &     1.0 &         2.0 &      2.0 &        2.25 &     3.0 \\
MultiMissão Leve &  1.375000 &     1 &       0.517549 &     1.0 &         1.0 &      1.0 &        2.00 &     2.0 \\
\bottomrule
\end{tabular}
\end{table}
    
       \item Sobre a questão que trata dos pontos de desconforto, foram construídos os gráficos da Figura \ref{fig:desconforto} % da Figura \ref{fig:pontosGeral} 
       e da \ref{fig:pontosFardamento} para verificar visualmente os pontos que mais incomodam os usuários.
\end{itemize}
       \figura{scale=0.5}{grafico_desconforto.png}{Pontos de desconforto indicados por fardamento}{fig:desconforto}

       %\figura{scale=0.5}{formulario_pontosGeral.png}{Nuvem de palavras com os pontos de desconforto}{fig:pontosGeral}

       \figura{scale=0.4}{formulario_pontosFardamento.png}{Nuvem de palavras com os pontos de desconforto indicados por fardamento}{fig:pontosFardamento}

 
    \subsection{ Análise Qualitativa:}
    As respostas abertas sobre os pontos de desconforto foram processadas em uma nuvem de palavras (Figura \ref{fig:descricaoFardamento}) para que fosse possível visualizar as palavras-chave mais recorrentes para cada fardamento.
    \figura{scale=0.4}{formulario_descricaoFardamento.png}{Nuvem de palavras com as palavras-chave da descrição de desconforto indicados por fardamento}{fig:descricaoFardamento}

\subsection{Conclusões}
    A escrever
