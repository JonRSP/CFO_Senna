\chapter{\textit{Functional Movement Screen}}

O \acrfull{FMS} é uma ferramenta padronizada de avaliação funcional amplamente utilizada para medir a qualidade do movimento
 em populações táticas, esportivas e clínicas. Foi desenvolvido com o objetivo de identificar padrões de movimento deficitários que podem aumentar
  o risco de lesões e comprometer o desempenho físico \cite{cook2006pre}.

\tab O \acrshort{FMS} avalia sete padrões de movimento fundamentais:

\lista{
	\textbf{Agachamento Profundo (\textit{Deep Squat})}: Mede mobilidade nos tornozelos, joelhos e quadris, além de estabilidade e mobilidade torácica.|
	\textbf{Passo em Barreiras (\textit{Hurdle Step})}: Avalia a coordenação e o equilíbrio dinâmico enquanto a perna livre realiza o movimento de passo.|
	\textbf{Avanço em Linha (\textit{In-Line Lunge})}: Testa a estabilidade do núcleo, mobilidade do quadril e coordenação em um plano de movimento linear.|
	\textbf{Mobilidade do Ombro (\textit{Shoulder Mobility})}: Mede a amplitude de movimento e simetria funcional do complexo do ombro.|
	\textbf{Elevação Ativa da Perna (\textit{Active Straight Leg Raise})}: Avalia a mobilidade dos isquiotibiais e estabilidade pélvica.|
	\textbf{Estabilidade de Tronco em Flexão (\textit{Trunk Stability Push-Up})}: Mede a estabilidade do núcleo em movimentos de empurrar.|
	\textbf{Estabilidade Rotacional (\textit{Rotary Stability})}: Avalia o controle motor e a estabilidade em movimentos de rotação.
	}

	\begin{centering}
		[Adicionar Foto montagem dos exercícios (ou exemplo de exercício)]
	\end{centering}

\tab Cada padrão de movimento é pontuado em uma escala de 0 a 3 \cite{cook2006pre, teyhen2012functional}: 
\lista{
	\textbf{0}: Ocorre dor durante o movimento.|
	\textbf{1}: O movimento é incompleto ou incorreto.|
	\textbf{2}: O movimento é completo, mas com compensações.|
	\textbf{3}: O movimento é realizado de maneira perfeita, sem compensações.
	}

\tab A soma total pode variar de 0 a 21, sendo escores abaixo de 14 associados a maior risco de lesões musculoesqueléticas \cite{bock2015use}.

\section{Relevância do \acrshort{FMS} em Populações Táticas}
O \acrshort{FMS} é particularmente valioso em populações como bombeiros, militares e policiais, onde tarefas operacionais demandam alta mobilidade funcional.
 Estudos demonstram que padrões de movimento deficitários, identificados pelo \acrshort{FMS}, podem ser agravados pelo uso de \acrshort{EPI}, aumentando o risco
  de lesões e diminuindo a eficiência dos movimentos \cite{orr2019impact}. 

\section{Limitações}
Apesar de sua utilidade, o \acrshort{FMS} tem limitações, incluindo a subjetividade na pontuação e a ausência de medidas diretas de força ou resistência.
 Portanto, é mais eficaz quando combinado com outras avaliações \cite{gribble2013intrarater}.