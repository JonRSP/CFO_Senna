\chapter{Formulário de Percepção Subjetiva}
\label{ap:formulario}
\section{Introdução}
Este documento apresenta a estrutura da análise subjetiva dos uniformes operacionais utilizados, considerando o impacto na mobilidade e os pontos de desconforto relatados pelos usuários.

\section{Questionário de Avaliação}

\subsection{Nome do Avaliado}
\vspace{2cm}
\subsection{Uniforme Avaliado}

    \begin{itemize}
        \item ( ) 5ºA
        \item ( ) MultiMissão
        \item ( ) MultiMissão Leve
    \end{itemize}


\subsection{Impacto do Uniforme nos Movimentos}
\textbf{Em uma escala de 1 a 5, indique o quanto o uniforme atrapalhou os movimentos realizados.}\\
(1 - Não atrapalhou, 5 - Atrapalhou muito)

\begin{center}
    \begin{tabular}{|c|c|c|c|c|}
        \hline
        1 & 2 & 3 & 4 & 5 \\
        \hline
    \end{tabular}
\end{center}

\subsection{Pontos de Desconforto}
\textbf{Se o uniforme causou algum desconforto, marque os pontos específicos onde isso ocorreu durante os movimentos realizados:}
\begin{itemize}
    \item ( ) Joelhos
    \item ( ) Coxas
    \item ( ) Virilha
    \item ( ) Glúteos
    \item ( ) Cintura
    \item ( ) Abdômen
    \item ( ) Peitoral
    \item ( ) Axilas
    \item ( ) Ombros
    \item ( ) Braços
    \item ( ) Antebraços
    \item ( ) Pescoço
\end{itemize}

\subsection{Descrição do Desconforto}
\textbf{Se o uniforme causou desconforto, descreva brevemente o tipo de sensação que você experimentou.:}
\vspace{2cm}

\section{Considerações Finais}
Os dados coletados por meio desta pesquisa serão utilizados para uma análise dos fardamentos operacionais do CBMSC.