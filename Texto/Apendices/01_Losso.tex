\chapter{O Fardamento 5ºA e Sua Influência no CBMSC}
\label{ap:losso}

\noindent \textbf{10 de Fevereiro de 2025}


\noindent \textbf{Entrevistado:} Coronel BM Losso

\section*{\textbf{Entrevista}}

\noindent \textbf{Cadete BM Senna:} \\ Qual era o contexto para o desenvolvimento do fardamento 5ºA?

\noindent \textbf{Coronel BM Losso:} \\ Em 2004, o Corpo de Bombeiros Militar de Santa Catarina (CBMSC) já estava separado da Polícia Militar desde 13 de junho de 2003. Naquele período, foi necessária a aprovação de várias legislações complementares para definir questões administrativas, como a permanência de bombeiros na nova instituição ou a possibilidade de retorno à Polícia Militar para aqueles que desejassem. Essa legislação também tratava da transferência de oficiais e praças da PM para o CBMSC, respeitando o quadro de vagas estabelecido na Lei de Fixação de Efetivo.

\noindent Após esse período de escolha, que durou cerca de seis meses e terminou em fevereiro de 2004, o comandante-geral da época, coronel Adilson Alcides de Oliveira, criou uma comissão para a criação do fardamento do CBMSC, composta por mim (coronel Diogo Bahia Losso), pelo coronel Alexandre Corrêa Dutra e pelo coronel Marcos de Oliveira.

\noindent O objetivo dessa comissão era desenvolver um uniforme próprio para os bombeiros, que historicamente sempre buscaram se diferenciar da Polícia Militar. Naquele momento, utilizávamos fardamento cáqui, mas já havíamos incorporado elementos distintos, como camiseta e cinto vermelhos, além de um calçado diferenciado, o borzeguim, em vez do coturno. Naquela época ainda não se utilizava, de maneira generalizada, EPI de combate a incêndio, por isso o tecido utilizado era brim 100\% algodão, pois esse material, ao contrário dos sintéticos, não aderiria à pele em caso de exposição ao fogo.

\noindent \textbf{Cadete BM Senna:} \\ Como foi a definição da cor do fardamento?

\noindent \textbf{Coronel BM Losso:} \\ Diante da necessidade de um novo uniforme, discutimos qual cor deveria ser adotada. A cor cáqui não era uma opção, pois poderia gerar confusão com a PM e representar riscos para os bombeiros que atuavam em áreas perigosas. 

\noindent Consideramos a cor cinza, mas logo descartamos essa opção, pois já era utilizada pelos Bombeiros Voluntários de Santa Catarina, e a adoção dessa cor pelo CBMSC poderia gerar conflitos institucionais. O coronel Adilson Alcides de Oliveira, então Comandante Geral, nos entregou uma farda operacional do Bombeiro da Polícia Militar de São Paulo, onde realizou o Curso de Especialização para Bombeiros, pedindo que usássemos aquela cor como referência. Esse tom, conhecido como "cinza bandeirante", tinha uma leve tonalidade azulada. A partir da solicitação do coronel Adilson Alcides de Oliveira, foi criada a cor “azul bandeirante” que é a utilizada no fardamento atual.

\noindent \textbf{Cadete BM Senna:} \\ Em algum momento, a cor laranja foi considerada?

\noindent \textbf{Coronel BM Losso:} \\ Não, pois houve uma experiência prévia negativa com o laranja 
na década de 1990. Os socorristas usavam naquela época uma farda de duas peças (calça e gandola) 
na cor laranja. Como não foi bem aceita, voltaram a usar a cor cáqui, com um colete laranja.
 Além disso, o laranja estava associado a empresas de limpeza urbana que operavam em diversas prefeituras
  do estado, o que tornava inviável sua adoção.

\noindent \textbf{Cadete BM Senna:} \\ Como foi a evolução do 5ºA ao longo do tempo?

\noindent \textbf{Coronel BM Losso:} \\ Inicialmente, o modelo do uniforme operacional seguiu o padrão da
 PM, com a gandola por dentro da calça e cinto vermelho. No entanto, em 2005-2006, ao iniciar a 
 distribuição do novo fardamento, identificamos um problema: o tecido brim desbotava rapidamente, tornando-se
  visualmente inadequado. O desgaste da cor azul era muito mais perceptível do que o da cor cáqui
   (usada antes da emancipação), impactando a imagem institucional.

\noindent Para resolver esse problema, migramos do brim 100\% algodão para o tecido "terbrim", composto por 67\% poliéster e 33\% algodão, com tecnologia ripstop, que proporcionava maior durabilidade e, também, maior resistência ao desbotamento. Com isso, foi necessário reforçar a necessidade do uso de EPIs para combate a incêndios, uma vez que o novo tecido sintético poderia aderir à pele em contato com chamas.

\noindent Removemos o bolso faca, acrescentamos os bolsos laterais e reforços nos joelhos e utilizamos
técnicas com o tecido em áreas estratégicas para melhorar a mobilidade. Desde o início,
também investimos na diferenciação do uniforme feminino, garantindo um corte mais adequado
ao corpo das bombeiras, algo que não existia na PM.

\noindent Outra alteração significativa, que ocorreu posteriormente à criação, foi no design da gandola,
que passou a ser usada por fora da calça, seguindo o modelo adotado pelo Exército e poroutros Corpos de Bombeiros estaduais.

\noindent \textbf{Cadete BM Senna:} \\ O manual de uniformes do CBMSC parece estar desatualizado, especialmente em relação ao
uniforme operacional. Por que isso ocorre?

\noindent \textbf{Coronel BM Losso:} \\ O regulamento de uniformes do CBMSC foi estabelecido por decreto, e qualquer alteração
formal requer outro decreto, envolvendo a Casa Civil e o governo do estado. No entanto, ao
longo dos anos, diversos ajustes foram feitos via portarias, sem a devida atualização do
regulamento oficial. Apesar da intenção inicial de manter um padrão estável, as mudanças
necessárias foram sendo implementadas de maneira fragmentada por diversos comandos.

\noindent Em um determinado período, existiu uma Coordenadoria de Uniformes, mas ela nunca chegou
a produzir uma atualização significativa ou a compilar essas portarias para transformar em um
novo decreto.

\noindent O regulamento utilizado como base, de forma geral, continha pouquíssimas imagens dos
uniformes, sendo que, na maioria das vezes, eram apenas fotos de policiais ou bombeiros
militares vestindo tais uniformes. O documento não possuía uma descrição minuciosa dos
trajes. Além disso, havia um problema relacionado ao uniforme operacional denominado 5ºA,
que abrangia a gandola e a calça, admitindo também o uso de jaqueta sobreposta e pullover,
isso pode gerar dúvidas sobre quais combinações são permitidas em um determinado evento,
por exemplo. No Exército, essa nomenclatura é utilizada de maneira diferente: ao adicionar
uma jaqueta ao uniforme, essa combinação passaria a ser denominada 5ºB, por exemplo.

\noindent \textbf{Cadete BM Senna:} \\ Qual é um grande desafio em relação ao fardamento operacional do CBMSC?

\noindent \textbf{Coronel BM Losso:} \\ Atualmente, um dos grandes desafios em relação ao uniforme do CBMSC é a adaptação às
diversas condições climáticas do estado. Santa Catarina possui microrregiões com
características muito distintas. Enquanto no interior as temperaturas são extremamente
elevadas, especialmente em regiões afastadas da brisa do mar, o litoral apresenta um clima
diferente. Florianópolis, por exemplo, tem uma variação menos extrema, no inverno não chega
a ser tão frio quanto na região da Serra, onde ocorrem temperaturas negativas e até neve.

\noindent Pensar em um uniforme único que atenda a todas essas variações é um grande desafio. Há
localidades onde, em um mesmo dia, ocorre uma grande amplitude térmica: o amanhecer pode
ser frio, o meio do dia quente, e à noite a temperatura volta a cair significativamente. Essa
oscilação dificulta a escolha de um material que proporcione conforto térmico adequado paratodas essas condições.

\noindent Além disso, a necessidade de um uniforme padronizado para toda a corporação deve ser
equilibrada com a funcionalidade e o conforto para o bombeiro em serviço. Encontrar essa
solução é uma tarefa complexa.

\noindent \textbf{Cadete BM Senna:} \\ Houve algum feedback da tropa sobre a mudança do fardamento?

\noindent \textbf{Coronel BM Losso:} \\ O primeiro feedback que recebemos foi negativo em relação ao desbotamento do tecido brim. No entanto, após a troca para o terbrim, sobre o uniforme operacional em si, não houve críticas relevantes. Pelo contrário, a mudança para a gandola por fora da calça foi bem recebida, melhorando a apresentação do efetivo.

\noindent Foram observados problemas em relação ao conforto térmico do fardamento com a mudança
de tecido 100\% algodão (brim) para a mescla atual (terbrim), o primeiro apresentando um
desempenho térmico melhor do que o segundo. Atualmente, existem tecidos inteligentes que
poderiam melhorar ainda mais a experiência dos bombeiros, mas a aquisição desse tipo de
material esbarra em questões orçamentárias e nos processos de licitação pública, que muitas
vezes impedem a compra de produtos com patentes específicas.

\noindent \textbf{Cadete BM Senna:} \\ Como a introdução do 5ºA impactou a identidade visual do CBMSC?

\noindent \textbf{Coronel BM Losso:} \\ A introdução do 5ºA ajudou a consolidar a identidade visual do CBMSC, diferenciando-o
definitivamente da Polícia Militar. O bombeiro passou a ter um uniforme próprio, e a mudança
foi bem recebida pela tropa. No início, devido à similaridade da cor azul, alguns bombeiros
foram confundidos com membros da Força Aérea Brasileira, mas essa questão foi rapidamente
superada com a consolidação do novo padrão.

\noindent A separação da PM trouxe a necessidade de afirmar a identidade dos bombeiros, e o uniforme
teve um papel fundamental nesse processo. A adoção do novo fardamento foi um marco para a
independência da corporação, tornando o CBMSC mais reconhecido e consolidando sua
imagem perante a sociedade.


