\chapter{O Fardamento 5ºA e Sua Influência no CBMSC}
\label{ap:losso}

\noindent \textbf{10 de Fevereiro de 2025}


\noindent \textbf{Entrevistado:} Coronel BM Losso

\section*{\textbf{Entrevista}}

\noindent \textbf{Cadete BM Senna:} \\ Qual era o contexto para o desenvolvimento do fardamento 5ºA?

\noindent \textbf{Coronel BM Losso:} \\ Em 2004, o Corpo de Bombeiros Militar de Santa Catarina (CBMSC) já estava separado da Polícia Militar desde 13 de junho de 2003. Naquele período, foi necessária a aprovação de várias legislações complementares para definir questões administrativas, como a permanência de bombeiros na nova instituição ou a possibilidade de retorno à Polícia Militar para aqueles que desejassem. Essa legislação também tratava da transferência de oficiais e praças da PM para o CBMSC, respeitando o quadro de vagas estabelecido na Lei de Fixação de Efetivo.

\noindent Após esse período de escolha, que durou cerca de seis meses e terminou em fevereiro de 2004, o comandante-geral da época, coronel Adilson Alcides de Oliveira, criou uma comissão para a criação do fardamento do CBMSC, composta por mim (coronel Diogo Bahia Losso), pelo coronel Alexandre Corrêa Dutra e pelo coronel Marcos de Oliveira.

\noindent O objetivo dessa comissão era desenvolver um uniforme próprio para os bombeiros, que historicamente sempre buscaram se diferenciar da Polícia Militar. Naquele momento, utilizávamos fardamento cáqui, mas já havíamos incorporado elementos distintos, como camiseta e cinto vermelhos, além de um calçado diferenciado, o borzeguim, em vez do coturno. Naquela época ainda não se utilizava, de maneira generalizada, EPI de combate a incêndio, por isso o tecido utilizado era brim 100\% algodão, pois esse material, ao contrário dos sintéticos, não aderiria à pele em caso de exposição ao fogo.

\noindent \textbf{Cadete BM Senna:} \\ Como foi a definição da cor do fardamento?

\noindent \textbf{Coronel BM Losso:} \\ Diante da necessidade de um novo uniforme, discutimos qual cor deveria ser adotada. A cor cáqui não era uma opção, pois poderia gerar confusão com a PM e representar riscos para os bombeiros que atuavam em áreas perigosas. 

\noindent Consideramos a cor cinza, mas logo descartamos essa opção, pois já era utilizada pelos Bombeiros Voluntários de Santa Catarina, e a adoção dessa cor pelo CBMSC poderia gerar conflitos institucionais. O coronel Adilson Alcides de Oliveira, então Comandante Geral, nos entregou uma farda operacional do Bombeiro da Polícia Militar de São Paulo, onde realizou o Curso de Especialização para Bombeiros, pedindo que usássemos aquela cor como referência. Esse tom, conhecido como "cinza bandeirante", tinha uma leve tonalidade azulada. A partir da solicitação do coronel Adilson Alcides de Oliveira, foi criada a cor “azul bandeirante” que é a utilizada no fardamento atual.

\noindent \textbf{Cadete BM Senna:} \\ Em algum momento, a cor laranja foi considerada?

\noindent \textbf{Coronel BM Losso:} \\ Não, pois houve uma experiência prévia negativa com o laranja na década de 1990. Os socorristas usavam naquela época uma farda de duas peças (calça e gandola) na cor laranja. Como não foi bem aceita, voltaram a usar a cor cáqui, com um colete laranja. Além disso, o laranja estava associado a empresas de limpeza urbana que operavam em diversas prefeituras do estado, o que tornava inviável sua adoção.

\noindent \textbf{Cadete BM Senna:} \\ Como foi a evolução do 5ºA ao longo do tempo?

\noindent \textbf{Coronel BM Losso:} \\ Inicialmente, o modelo do uniforme operacional seguiu o padrão da PM, com a gandola por dentro da calça e cinto vermelho. No entanto, em 2005-2006, ao iniciar a distribuição do novo fardamento, identificamos um problema: o tecido brim desbotava rapidamente, tornando-se visualmente inadequado. O desgaste da cor azul era muito mais perceptível do que o da cor cáqui (usada antes da emancipação), impactando a imagem institucional.

\noindent Para resolver esse problema, migramos do brim 100\% algodão para o tecido "terbrim", composto por 67\% poliéster e 33\% algodão, com tecnologia ripstop, que proporcionava maior durabilidade e, também, maior resistência ao desbotamento. Com isso, foi necessário reforçar a necessidade do uso de EPIs para combate a incêndios, uma vez que o novo tecido sintético poderia aderir à pele em contato com chamas.

\noindent \textbf{Cadete BM Senna:} \\ Houve algum feedback da tropa sobre a mudança do fardamento?

\noindent \textbf{Coronel BM Losso:} \\ O primeiro feedback que recebemos foi negativo em relação ao desbotamento do tecido brim. No entanto, após a troca para o terbrim, sobre o uniforme operacional em si, não houve críticas relevantes. Pelo contrário, a mudança para a gandola por fora da calça foi bem recebida, melhorando a apresentação do efetivo.

\noindent \textbf{Cadete BM Senna:} \\ Como a introdução do 5ºA impactou a identidade visual do CBMSC?

\noindent \textbf{Coronel BM Losso:} \\ A introdução do 5ºA ajudou a consolidar a identidade visual do CBMSC, diferenciando-o definitivamente da Polícia Militar. O bombeiro passou a ter um uniforme próprio, e a mudança foi bem recebida pela tropa.


