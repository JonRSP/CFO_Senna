\chapter{Introdução}
% ----------------------------------------------------------



	O \acrfull{CBMSC} desempenha um papel crucial na proteção da sociedade catarinense, atuando em 
	diversas áreas como combate a incêndios, resgates, salvamentos e atendimentos pré-hospitalares. 
	A complexidade e a diversidade das missões exigem que os bombeiros militares estejam preparados 
	para enfrentar diferentes desafios, desde ocorrências em ambientes urbanos até situações em 
	áreas remotas e de difícil acesso.  Nesse contexto, o fardamento operacional se torna um 
	elemento fundamental para garantir a segurança e o bem-estar dos bombeiros, influenciando 
	diretamente na sua capacidade de atuação.

	\section{Problema}

		A atividade de busca terrestre, em particular, apresenta desafios específicos para os bombeiros militares.
		As buscas podem ocorrer em terrenos acidentados, com vegetação densa, em condições climáticas adversas e por longos períodos,
		demandando um fardamento que ofereça, além de resistência mecânica, conforto térmico e mobilidade.  
		A escolha do uniforme adequado para essa atividade é crucial para garantir a eficiência da busca e a segurança dos bombeiros envolvidos.

		O \acrshort{CBMSC} dispõe de diferentes tipos de uniformes operacionais, cada um com características específicas para atender
		às demandas de diferentes atividades.  Entre os uniformes utilizados na atividade de \acrfull{BTR}, destacam-se o 5ºA e o Multimissão.
		O 5ºA é o fardamento padrão utilizado pelo \acrshort{CBMSC} nas atividades cotidianas, como atendimento de ocorrências,
		treinamentos e serviços administrativos. É composto por gandola e calça de \textit{terbrim} com ripstop, um tecido mais leve e 
		resistente, e camiseta de algodão, proporcionando praticidade para o uso diário.

		O uniforme Multimissão foi desenvolvido para atender às necessidades de bombeiros que atuam em diferentes áreas, como resgate veicular,
	 	combate a incêndio florestal e salvamento em altura.

		Considerando a importância de um fardamento adequado para a atividade de \acrshort{BTR}, este trabalho propõe a análise e comparação do
	 	uniforme 5ºA, do Multimissão e de uma proposta de EPI Leve, a fim de se verificar com tal pesquisa, \textbf{qual uniforme oferece o melhor desempenho 
	 	em termos de conforto térmico e mobilidade para a atividade de \acrlong{BTR}?}

	\section{Objetivos}

		\subsection{Objetivo Geral}
			\begin{itemize}
				\item Comparar os uniformes operacionais aplicáveis à atividade de \acrlong{BTR}.
			\end{itemize}
		\subsection{Objetivos Específicos}
			\begin{itemize}
				\item Avaliar a mobilidade permitida pelos uniformes operacionais utilizados na área de \acrlong{BTR} por meio do \acrfull{FMS};
				\item Avaliar a temperatura em certos pontos dos uniformes operacionais utilizados para \acrlong{BTR} durante a execução de atividades físicas por meio de sensores de temperatura;
				\item Apontar o uniforme mais adequado para aplicação na área de \acrlong{BTR}.
			\end{itemize}
\section{Justificativa}
	
	A atividade de busca terrestre apresenta peculiaridades que exigem uma análise detalhada dos 
	fardamentos utilizados. Terrenos acidentados, vegetação densa, condições climáticas adversas 
	e longas jornadas demandam uniformes que conciliem resistência, mobilidade e conforto térmico.
	
	Considerando a diversidade de riscos presentes em uma operação de busca, o fardamento operacional é um componente estratégico para
	 a segurança e eficiência dos bombeiros, influenciando diretamente sua capacidade de atuação em situações adversas.
	
	Atualmente, os uniformes 5ºA e \acrlong{MM}, amplamente utilizados pelo \acrshort{CBMSC}, possuem características idealizadas para diferentes contextos,
	  entretanto, ainda não há um estudo sobre características objetivas que influenciam diretamente no seu uso pelos bombeiros militares.
	
	  A análise dos fardamentos operacionais a fim de se definir o melhor uniforme para as atividades de busca terrestre é fundamental 
	  para o desempenho eficiente e seguro das equipes de resgate. Este estudo é justificado pela necessidade de alinhar os 
	  \acrfull{EPI} com os desafios específicos enfrentados nas operações de busca, além de se tratar de proposta de uma metodologia
	   para análise objetiva de fardamentos da corporação.

%\newpage