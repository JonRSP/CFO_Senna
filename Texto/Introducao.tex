\chapter{Introdução}
% ----------------------------------------------------------



	    O \acrfull{CBMSC} atua em uma ampla gama de cenários, desde ambientes urbanos até 
		regiões remotas e de difícil acesso, o que exige que seus bombeiros portem equipamentos 
		compatíveis com as demandas e riscos operacionais de cada missão. Nesse contexto, o 
		fardamento operacional deixa de ser apenas um elemento de identidade institucional, passando 
		a representar um componente estratégico para a segurança, o conforto e o desempenho dos 
		militares em serviço.

    	Dada a importância do fardamento como elemento operacional, torna-se imprescindível o 
		desenvolvimento e utilização de metodologias objetivas para sua avaliação técnica. 
		Para que as decisões sobre esse tipo de equipamento sejam mais tecnicamente fundamentadas e 
		alinhadas às especificidades de cada atividade, é necessário que as metodologias adotadas 
		permitam mensurar, comparar e validar as diferentes opções disponíveis, considerando as 
		particularidades das diversas áreas de atuação da corporação e promovendo maior transparência 
		e coerência nas escolhas institucionais voltadas à padronização dos equipamentos operacionais.

	\section{Problema}

		A atividade de busca terrestre impõe desafios operacionais singulares aos bombeiros militares,
		 especialmente em contextos que envolvem terrenos acidentados, vegetação densa, condições climáticas 
		 adversas e longos períodos de atuação, demandando um fardamento que ofereça, além de resistência mecânica, conforto térmico e mobilidade.  
		A escolha do uniforme adequado para essa atividade é crucial para garantir a eficiência da busca e a segurança dos bombeiros envolvidos.

		O \acrshort{CBMSC} dispõe de diferentes tipos de uniformes operacionais, cada um com características específicas para atender
		às demandas de diferentes atividades.  Entre os uniformes utilizados na atividade de \acrfull{BTR}, destacam-se o 5ºA e o Multimissão.
		O 5ºA é o fardamento padrão utilizado pelo \acrshort{CBMSC} nas atividades cotidianas, como atendimento de ocorrências,
		treinamentos e serviços administrativos. É composto por gandola e calça de \textit{terbrim} com \textit{ripstop}, um tecido mais leve e 
		resistente, e camiseta de algodão, proporcionando praticidade para o uso diário.

		O \acrshort{EPI} \acrlong{MM} foi desenvolvido para atender às necessidades de bombeiros que atuam 
		em diferentes áreas e institucionalizado na corporação em 2023 para as áreas de resgate 
		veicular, combate a incêndio florestal, salvamento em altura, corte de árvores, 
		deslizamentos, busca e resgate em estruturas colapsadas, busca terrestre, cinotecnia 
		e atendimento pré-hospitalar \cite{res16}. 
		
		É necessário ressaltar que a escolha do fardamento mais adequado para as atividades desempenhadas pelo
		\acrshort{CBMSC} é muitas vezes baseada em testes sem padronização e não replicáveis, sem 
		necessariamente se basear em critérios comparativos sistematizados e validados.

		Nesse contexto nota-se a ausência de uma metodologia padronizada, quantitativa e multifatorial que 
		permita a avaliação objetiva e replicável de uniformes operacionais. Essa lacuna metodológica 
		pode afetar a padronização de práticas institucionais, a segurança dos bombeiros e a eficiência 
		das operações em ambientes hostis. Diante disso, este trabalho propõe uma metodologia estruturada, 
		que combina a análise funcional (por meio do \acrfull{FMS}), desempenho térmico (via sensores 
		desenvolvidos) e a percepção subjetiva dos usuários, visando fornecer subsídios técnicos para 
		a escolha de fardamentos e equipamentos de proteção individual com base em evidências.

		Considerando o contexto operacional específico da \acrlong{BTR}, este trabalho propõe uma metodologia 
		para análise e comparação de uniformes e equipamentos de proteção individuais aplicáveis à 
		atividade, sendo objetos o uniforme 5ºA, o Multimissão e uma proposta de \acrshort{EPI} Leve, 
		a fim de se verificar com tal pesquisa, \textbf{qual uniforme oferece o melhor desempenho 
	 	em termos de conforto térmico e mobilidade para a atividade de \acrlong{BTR}?}

	\section{Objetivos}

		\subsection{Objetivo Geral}
			\begin{itemize}
				\item Comparar, por meio da metodologia proposta, os uniformes operacionais 
				aplicáveis à atividade de \acrlong{BTR}.
			\end{itemize}
		\subsection{Objetivos Específicos}
			\begin{itemize}
				\item Aplicar o protocolo \acrfull{FMS} para avaliar a mobilidade funcional dos uniformes operacionais utilizados na área de \acrlong{BTR};
				\item Monitorar o comportamento térmico dos uniformes em pontos estratégicos durante a execução de atividades físicas, por meio de sensores desenvolvidos;
				\item Analisar a percepção subjetiva dos usuários quanto ao conforto e restrição de movimento por meio de questionários estruturados;
				\item Comparar os resultados obtidos entre os uniformes 5ºA, Multimissão e \acrshort{EPI} Leve, a fim de identificar a opção mais adequada à atividade de \acrlong{BTR}.
			\end{itemize}

\section{Justificativa}
	
	A atividade de busca terrestre apresenta peculiaridades que exigem uma análise detalhada dos 
	fardamentos utilizados. Terrenos acidentados, vegetação densa, condições climáticas adversas 
	e longas jornadas demandam uniformes que conciliem resistência, mobilidade e conforto térmico.
	
	Considerando a diversidade de riscos presentes em uma operação de busca, o fardamento operacional é um componente estratégico para
	 a segurança e eficiência dos bombeiros, influenciando diretamente sua capacidade de atuação em situações adversas.
	
	Atualmente, os uniformes 5ºA e \acrlong{MM}, amplamente utilizados pelo \acrshort{CBMSC}, possuem 
	características idealizadas para diferentes contextos,
	  entretanto, Apesar de seus atributos permitirem a aplicação em diferentes áreas, ainda não existem critérios objetivos que 
	  orientem sua adoção em atividades específicas. A atribuição dos uniformes às funções operacionais 
	  permanece, portanto, baseada em critérios subjetivos.
	
	  Diante desse cenário, este estudo propõe o desenvolvimento de uma metodologia objetiva e estruturada
	   para avaliação de fardamentos operacionais no âmbito do \acrshort{CBMSC}. A abordagem integra 
	   critérios funcionais, térmicos e perceptivos de forma sistematizada, oferecendo subsídios técnicos para 
	   a comparação entre diferentes modelos de uniformes. A atividade de \acrlong{BTR} foi escolhida 
	   como cenário de aplicação por representar um contexto operacional exigente, ideal para validar a 
	   efetividade da metodologia proposta. Ao alinhar rigor técnico com demandas práticas da corporação, 
	   o estudo propõe uma base metodológica para avaliação de fardamentos. Trata-se de uma iniciativa 
	   voltada à aumentar a transparência, replicabilidade e fundamentação técnica do processo de decisão
	   sobre uniformes.
%\newpage